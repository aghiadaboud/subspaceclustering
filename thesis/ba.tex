%%%%%%%%%%%%%%%%%%%%%%%%%%%%%%%%%%%%%%%%%%%%%%%%%%%%%%%%%%%%%%%%%%%%%%%%
% Uni Duesseldorf
% Lehrstuhl fuer Datenbanken und Informationssysteme
% Vorlage fuer Bachelor-/Masterarbeiten
% Optimiert fuer den Original-Latex-Kompiler LATEX.EXE (LaTeX=>PS=>PDF)
%%%%%%%%%%%%%%%%%%%%%%%%%%%%%%%%%%%%%%%%%%%%%%%%%%%%%%%%%%%%%%%%%%%%%%%%
% Ueberarbeitung für pdflatex (LaTeX=>PDF)
%%%%%%%%%%%%%%%%%%%%%%%%%%%%%%%%%%%%%%%%%%%%%%%%%%%%%%%%%%%%%%%%%%%%%%%%
% Vorlage Changelog:
% 10.09.2015 (Matthias Liebeck): Nummerierung des Inhaltsverzeichnis nun römisch, Beispiel für einen Anhang eingebaut, \raggedbottom hinter sections eingefügt
% 11.07.2018 (Matthias Liebeck): Ersetzung des Bibliographiestils, Einsatz von Biber
% 04.09.2018 (Matthias Liebeck):
%   * Bibtex: unnötige Bibtexfelder beim Rendern ausblenden (thx @ Markus Brenneis)
%   * ngerman: "et al." im BibTeX für drei oder mehr Autoren
%   * Neuer Befehl \sectionforcestartright: Sections immer rechts beginnen (thx @ Philipp Grawe)
%   * ngerman: Deutsche Anführungszeichen im Literaturverzeichnis (thx @ Markus Brenneis)
%   * ngerman: Deutsche Anführungszeichen im Literaturverzeichnis (thx @ Markus Brenneis)
% 16.10.2018 (Matthias Liebeck): Zwei fixes an \sectionforcestartright (thx @ Markus Brenneis)
%%%%%%%%%%%%%%%%%%%%%%%%%%%%%%%%%%%%%%%%%%%%%%%%%%%%%%%%%%%%%%%%%%%%%%%%
%%%% BEGINN EINSTELLUNG FUER DIE ARBEIT. UNBEDINGT ERFORDERLICH! %%%%%%%
%%%%%%%%%%%%%%%%%%%%%%%%%%%%%%%%%%%%%%%%%%%%%%%%%%%%%%%%%%%%%%%%%%%%%%%%
% Geben Sie Ihren Namen hier an:

\newcommand{\bearbeiter}{Aghied Aboud}

% Geben Sie hier den Titel Ihrer Arbeit an:
\newcommand{\titel}{Implementation and Comparison of Subspace Clustering Methods}

% Geben Sie das Datum des Beginns und Ende der Bachelorarbeit ein:
\newcommand{\beginndatum}{03. August 2022}
\newcommand{\abgabedatum}{03.~November~2022}

% Geben Sie die Namen des Erst- und Zweitgutachters an:
\newcommand{\erstgutachter}{Prof. Dr.~Stefan Conrad}
\newcommand{\zweitgutachter}{Prof. Dr.~Melanie Schmidt}

% Falls Sie die Arbeit zweiseitig ausdrucken wollen,
% benutzen Sie die folgende Zeile mit
% \AN fuer zweiseitigen Druck
% \AUS fuer einseitigen Druck
\newcommand{\zweiseitig}{\AN}
% true fuer biber, false fuer klassischen Zitierstil
%\newcommand{\biber}{false}
\newcommand{\biber}{true}

% Falls Sections immer rechts beginnen sollen. Gerade für Masterarbeiten
% interessant. Bei kurzen Bachelorarbeiten eher weniger zu verwenden.
\newcommand{\sectionforcestartright}{false}
%\newcommand{\sectionforcestartright}{true}

% Falls die Arbeit in englischer Sprache verfasst
% werden soll, dann benutzen Sie die folgende Zeile mit
% englisch fuer englische Sprache
% deutsch fuer deutsche Sprache
\newcommand{\sprache}{englisch}

% Hier wird eingestellt, ob es sich bei der Arbeit um eine Bachelor-
% oder Masterarbeit handelt (unpassendes auskommentieren!):
\newcommand{\arbeit}{Bachelorarbeit}
%~ \newcommand{\arbeit}{Masterarbeit}


%%%%%%%%%%%%%%%%%%%%%%%%%%%%%%%%%%%%%%%%%%%%%%%%%%%%%%%%%%%%%%%%%%%%%%%%
%%%% ENDE EINSTELLUNGEN %%%%%%%%%%%%%%%%%%%%%%%%%%%%%%%%%%%%%%%%%%%%%%%%
%%%%%%%%%%%%%%%%%%%%%%%%%%%%%%%%%%%%%%%%%%%%%%%%%%%%%%%%%%%%%%%%%%%%%%%%

% Die folgende Zeile NICHT EDITIEREN oder loeschen


%%%%%%%%%%%%%%%%%%%%%%%%%%%%%%%%%%%%%%%%%%%%%%%%%%%%%%%%%%%
% Obere Titelmakros. Editieren Sie diese Datei nur, wenn
% Sie sich ABSOLUT sicher sind, was Sie da tun!!!
% (Z.B. zum Abaendern der BA-Vorlage in eine MA-Vorlage)
% Uni Duesseldorf
% Lehrstuhl fuer Datenbanken und Informationssysteme
% Version 2.2 - 2.3.2010
%%%%%%%%%%%%%%%%%%%%%%%%%%%%%%%%%%%%%%%%%%%%%%%%%%%%%%%%%%%
\newcommand{\AN}{twoside}
\newcommand{\AUS}{}


%\newcommand{\englisch}{}
%\newcommand{\deutsch}{\usepackage[german]{babel}}

%% Die folgenden auskommentierten Optionen dienen der automatischen
%% Erkennung des Latex-Kompilers und dem Setzen der davon abhängigen
%% Einstellungen. Bei Problem z.B. mit dem Einbinden von verschiedenen
%% Grafiktypen bei Verwendung von PdfLatex oder Latex, einfach die
%% verschiedenen \usepackage(s) ausprobieren. (Mit diesen Einstellungen
%% funktionierte diese Vorlage bei der Verwenundg von latex.exe als
%% Kompiler bei den meisten Studierenden.)

%\newif\ifpdf \ifx\pdfoutput\undefined
%\pdffalse % we are not running pdflatex
%\else
%\pdfoutput=1 % we are running pdflatex
%\pdfcompresslevel=9 % compression level for text and image;
%\pdftrue \fi

\documentclass[11pt,a4paper, \zweiseitig]{article}
\usepackage{ifthen}

\usepackage[dvipsnames]{xcolor}
\usepackage{todonotes}

%\usepackage[iso]{umlaute}
\usepackage[utf8]{inputenc}
\usepackage{palatino} % palatino Schriftart
%\usepackage{makeidx} % um ein Index zu erstellen
\usepackage[nottoc]{tocbibind}
\usepackage[T1]{fontenc} %fuer richtige Trennung bei Umlauten
\usepackage{fancybox} % fuer die Rahmen
\usepackage{shortvrb}
\usepackage{url}
\usepackage{xcolor}
\usepackage[colorlinks,citecolor=blue,linkcolor=black]{hyperref} %anklickbares Inhaltsverzeichnis

\ifthenelse{\boolean{\biber}}{
  % only needed for biber
  \usepackage[style=authoryear,natbib=true,backend=biber,mincitenames=1,maxcitenames=2,maxbibnames=99,uniquelist=false,dashed=false]{biblatex}

  % https://tex.stackexchange.com/a/334703/8850
  \AtEveryBibitem{%
    \clearfield{issn}
    \clearfield{isbn}
    \clearfield{doi}
    \clearfield{location}
    \clearlist{location}
    \clearlist{address}

    \ifentrytype{online}{}{% Remove url except for @online
      \clearfield{url}
    }
  }
}
{}%no else

% Falls es bei \citet ein Komma zwischen Name und Jahr gibt:
% https://tex.stackexchange.com/questions/312539/unwanted-comma-between-author-and-year-using-citet-command
% (thx @ Markus Brenneis)
%\DeclareDelimFormat[cbx@textcite]{nameyeardelim}{\addspace}



\ifthenelse{\equal{\sprache}{deutsch}}{
  \usepackage[ngerman]{babel}
  % Bibtex u.a -> et al.
  \ifthenelse{\boolean{\biber}}{
    \DefineBibliographyStrings{ngerman}{
      andothers = {{et\,al\adddot}},
    }
    \newcommand{\references}{Literatur}
  }
  {} % do nothing when not using biber
  \usepackage[autostyle, german=quotes]{csquotes} % Deutsche Anführungszeichen im Literaturverzeichnis (thx @ Markus Brenneis)

}{ \newcommand{\references}{References}}

\usepackage{a4wide} % ganze A4 Weite verwenden



%\ifpdf
%\usepackage[pdftex,xdvi]{graphicx}
%\usepackage{thumbpdf} %thumbs fuer Pdf
%\usepackage[pdfstartview=FitV]{hyperref} %anklickbares Inhaltsverzeichnis
%\else
%\usepackage[dvips,xdvi]{graphicx}
\usepackage{graphicx}
%%%%%%%
\usepackage{algorithm2e}
\usepackage[all]{hypcap}
\usepackage{mathtools}
\usepackage{tikz}
\usepackage{multirow}
\usepackage{float}
\usepackage{amssymb}
\usepackage{pifont}
%\fi

\newcommand{\redt}[1] {
  \textcolor{red}{#1}}

\newcommand{\oranget}[1] {
  \textcolor{orange}{#1}}

\newcommand{\purplet}[1] {
  \textcolor{purple}{#1}}

%%%%%%%%%%%%%%%%%%%%%%% Massangaben fuer die Arbeit %%%%%%%%%%%%%%%
\setlength{\textwidth}{15cm}

\setlength{\oddsidemargin}{35mm}
\setlength{\evensidemargin}{25mm}

\addtolength{\oddsidemargin}{-1in}
\addtolength{\evensidemargin}{-1in}

\ifthenelse{\boolean{\biber}}{\addbibresource{references.bib}}{}

%\makeindex
\begin{document}
%\setcounter{secnumdepth}{4} %Nummerieren bis in die 4. Ebene
%\setcounter{tocdepth}{4} %Inhaltsverzeichnis bis zur 4. Ebene

\pagestyle{headings}

\sloppy % LaTeX ist dann nicht so streng mit der Silbentrennung
%~ \MakeShortVerb{\§}

\parindent0mm
\parskip0.5em


{
\textwidth170mm
\oddsidemargin30mm
\evensidemargin30mm
\addtolength{\oddsidemargin}{-1in}
\addtolength{\evensidemargin}{-1in}

\parskip0pt plus2pt

% Die Raender muessen eventuell fuer jeden Drucker individuell eingestellt
% werden. Dazu sind die Werte fuer die Abstaende `\oben' und `\links' zu
% aendern, die von mir auf jeweils 0mm eingestellt wurden.

%\newlength{\links} \setlength{\links}{10mm}  % hier abzuaendern
%\addtolength{\oddsidemargin}{\links}
%\addtolength{\evensidemargin}{\links}

\begin{titlepage}
\vspace*{-1.5cm}
\raisebox{17mm}{
    \begin{minipage}[t]{70mm}
        \begin{center}
            %\selectlanguage{german}
            {\Large INSTITUT FÜR INFORMATIK\\}
            {\normalsize
                Datenbanken und Informationssysteme\\
            }
            \vspace{3mm}
            {\small Universitätsstr. 1 \hspace{5ex} D--40225 Düsseldorf\\}
        \end{center}
    \end{minipage}
}
\hfill
\raisebox{7mm}{
    \includegraphics[width=130pt]{bilder/HHU_Logo}}
\vspace{14em}

% Titel
\begin{center}
    \baselineskip=55pt
    \textbf{\huge \titel}
    \baselineskip=0 pt
\end{center}

%\vspace{7em}

\vfill

% Autor
\begin{center}
    \textbf{\Large
        \bearbeiter
    }
\end{center}

\vspace{35mm}

% Prüfungsordnungs-Angaben
\begin{center}
%\selectlanguage{german}

%%%%%%%%%%%%%%%%%%%%%%%%%%%%%%%%%%%%%%%%%%%%%%%%%%%%%%%%%%%%%%%%%%%%%%%%%
% Ja, richtig, hier kann die BA-Vorlage zur MA-Vorlage gemacht werden...
% (nicht mehr nötig!)
%%%%%%%%%%%%%%%%%%%%%%%%%%%%%%%%%%%%%%%%%%%%%%%%%%%%%%%%%%%%%%%%%%%%%%%%%
{\Large \arbeit}

\vspace{2em}
\ifthenelse{\equal{\sprache}{deutsch}}{
    \begin{tabular}[t]{ll}
    Beginn der Arbeit:& \beginndatum \\
    Abgabe der Arbeit:& \abgabedatum \\
    Gutachter:         & \erstgutachter \\
    & \zweitgutachter \\
}{
    \begin{tabular}[t]{ll}
    Date of issue:& \beginndatum \\
    Date of submission:& \abgabedatum \\
    Reviewers:         & \erstgutachter \\
    & \zweitgutachter \\
}
\end{tabular}
\end{center}

\end{titlepage}

}

%%%%%%%%%%%%%%%%%%%%%%%%%%%%%%%%%%%%%%%%%%%%%%%%%%%%%%%%%%%%%%%%%%%%%
\clearpage
\begin{titlepage}
    ~                % eine leere Seite hinter dem Deckblatt
\end{titlepage}
%%%%%%%%%%%%%%%%%%%%%%%%%%%%%%%%%%%%%%%%%%%%%%%%%%%%%%%%%%%%%%%%%%%%%
\clearpage
\begin{titlepage}
    \vspace*{\fill}

    \section*{Erklärung}

    %%%%%%%%%%%%%%%%%%%%%%%%%%%%%%%%%%%%%%%%%%%%%%%%%%%%%%%%%%%
    % Und hier ebenfalls ggf. BA durch MA ersetzen...
    % (Auch nicht mehr nötig!)
    %%%%%%%%%%%%%%%%%%%%%%%%%%%%%%%%%%%%%%%%%%%%%%%%%%%%%%%%%%%

    Hiermit versichere ich, dass ich diese \arbeit{}
    selbstständig verfasst habe. Ich habe dazu keine anderen als die
    angegebenen Quellen und Hilfsmittel verwendet.

    \vspace{25 mm}

    \begin{tabular}{lc}
        Düsseldorf, den \abgabedatum \hspace*{2cm} & \underline{\hspace{6cm}} \\
                                                   & \bearbeiter
    \end{tabular}

    \vspace*{\fill}
\end{titlepage}

%%%%%%%%%%%%%%%%%%%%%%%%%%%%%%%%%%%%%%%%%%%%%%%%%%%%%%%%%%%%%%%%%%%%%
% Leerseite bei zweiseitigem Druck
%%%%%%%%%%%%%%%%%%%%%%%%%%%%%%%%%%%%%%%%%%%%%%%%%%%%%%%%%%%%%%%%%%%%%

\ifthenelse{\equal{\zweiseitig}{twoside}}{\clearpage\begin{titlepage}
        ~\end{titlepage}}{}

%%%%%%%%%%%%%%%%%%%%%%%%%%%%%%%%%%%%%%%%%%%%%%%%%%%%%%%%%%%%%%%%%%%%%
\clearpage
\begin{titlepage}

    %%% Die folgende Zeile nicht ändern!
\section*{\ifthenelse{\equal{\sprache}{deutsch}}{Zusammenfassung}{Abstract}}\raggedbottom
%%% Zusammenfassung:

Subspace clustering has been developed to meet the requirements for clustering high-dimensional data at a time when traditional clustering methods were not able to solve the problem properly. It is a powerful method that mines hidden clusters in subspaces of the original feature space by combining two tasks, subspace search and clustering.

Subspace clustering has been successfully applied in fields like gene expression analysis, where some clusters can only be present in locally relevant subsets of dimensions.

There are many subspace clustering algorithms, but in this thesis we will discuss only SUBCLU \citep{subclu} and FIRES \citep{fires} in detail, two density-based subspace clustering algorithms. We will illustrate an example of clustering done by our implementation of both algorithms and we will evaluate their scalability and the quality of their clustering results.

In addition, we will talk about the reasons for which subspace clustering was developed. Then we will mention some applications where it can be employed and some alternatives for the method. We will also list all different classes of subspace clustering approaches and explain in which aspects these approaches can differ.


    %%%%%%%%%%%%%%%%%%%%%%%%%%%%%%%%%%%%%%%%%%%%%%%%
    % Untere Titelmakros. Editieren Sie diese Datei nur, wenn Sie sich
    % ABSOLUT sicher sind, was Sie da tun!!!
    %%%%%%%%%%%%%%%%%%%%%%%%%%%%%%%%%%%%%%%%%%%%%%%
    \vspace*{\fill}
\end{titlepage}

%%%%%%%%%%%%%%%%%%%%%%%%%%%%%%%%%%%%%%%%%%%%%%%%%%%%%%%%%%%%%%%%%%%%%
% Leerseite bei zweiseitigem Druck
%%%%%%%%%%%%%%%%%%%%%%%%%%%%%%%%%%%%%%%%%%%%%%%%%%%%%%%%%%%%%%%%%%%%%
\ifthenelse{\equal{\zweiseitig}{twoside}}
{\clearpage\begin{titlepage}~\end{titlepage}}{}
%%%%%%%%%%%%%%%%%%%%%%%%%%%%%%%%%%%%%%%%%%%%%%%%%%%%%%%%%%%%%%%%%%%%%
\clearpage \setcounter{page}{1}
\pagenumbering{roman}
\setcounter{tocdepth}{2}
\tableofcontents

%\enlargethispage{\baselineskip}
\clearpage
%%%%%%%%%%%%%%%%%%%%%%%%%%%%%%%%%%%%%%%%%%%%%%%%%%%%%%%%%%%%%%%%%%%%%
% Leere Seite, falls Inhaltsverzeichnis mit ungerader Seitenzahl und
% doppelseitiger Druck
%%%%%%%%%%%%%%%%%%%%%%%%%%%%%%%%%%%%%%%%%%%%%%%%%%%%%%%%%%%%%%%%%%%%%
\ifthenelse{ \( \equal{\zweiseitig}{twoside} \and \not \isodd{\value{page}} \)}
{\pagebreak \thispagestyle{empty} \cleardoublepage}{\clearpage}


% Kapitel soll bei doppelseitigem Druck immer auf der rechten (ungeraden) Seite anfangen (thx @ Philipp Grawe)
% https://tex.stackexchange.com/a/223387
\ifthenelse{\boolean{\sectionforcestartright}}
{\let\oldsection\section % Store \section in \oldsection
    \renewcommand{\section}{\cleardoublepage\oldsection}}
{}
\pagenumbering{arabic}
\setcounter{page}{1}

%%%%%%%%%%%%%%%%%%%%%%%%%%%%%%%%%%%%%%%%%%%%%%%%%%%%%%%%%%%%%%%%%%%%%%%%
%%%% BEGINN TEXTTEIL %%%%%%%%%%%%%%%%%%%%%%%%%%%%%%%%%%%%%%%%%%%%%%%%%%%
%%%%%%%%%%%%%%%%%%%%%%%%%%%%%%%%%%%%%%%%%%%%%%%%%%%%%%%%%%%%%%%%%%%%%%%%

%%%%%%%%%%%%%%%%%%%%%%%%%%%%%%%%%%%%%%%%%%%%%%%%%%%%%%%%%%%%%%%%%%%%%%%%
% Text entweder direkt hier hinein schreiben oder, im Sinne der
% besseren Uebersichtlich- und Bearbeitbarkeit mittels \input die
% einzelnen Textteile hier einbinden.
%%%%%%%%%%%%%%%%%%%%%%%%%%%%%%%%%%%%%%%%%%%%%%%%%%%%%%%%%%%%%%%%%%%%%%%%

\section{Introduction}\raggedbottom

\subsection{Background}
With the growth in the number of data sources and technologies used to register data over time, a large amount of big data are produced nowadays, which then are used for developing researches, solving problems and gaining new knowledge. The process of analyzing data sets and discovering patterns to achieve such goals mentioned above is called data mining. One of data mining tasks that aims to find similar groups and structures in the data is clustering. In fact, clustering is a well researched technique in data science, but as recorded data have become bigger and bigger in time, classical clustering methods often fail to effectively examine and group the data correctly. With large data we mean data with big feature space where each feature can have a range of possible values. Thus, the need for new clustering tools such as subspace clustering methods has grown.

With the help of subspace clustering algorithms, we aim to better process high dimensional data to better extract information contained within it, detect meaningful clusters possibly hidden in subspaces of the complete feature space and efficiently detect overlapping subspace clusters if the algorithm is capable.
 
One important reason for subspace clustering is that large spaces often consist of noisy, irrelevant and highly correlated dimensions that negatively affects the quality of a full-space clustering. Usually, dimensions in a large space are not concurrently relevant for all given clusters and therefore only need to be considered when they are meaningful. Subspace clustering methods work on solving this local feature relevance problem by searching for relevant subsets of features or important correlation of features for clusters, since clusters might also exist in arbitrarily oriented subspaces.
In addition, the more features represent an observation, the more sparse it becomes. This phenomenon is one of various phenomena known as the curse of dimensionality that states how the volume of a space rapidly increases as the dimensionality increases. Making it harder for distance functions to precisely measure similarity between observations in high dimensional feature spaces as objects appear to be equidistant thus dissimilar. For that, clustering high dimensional data is a challenging task. Besides that, partitioning the data can change from one subspace to another, so for a user it would be interesting to see how data objects variously group in different subspaces.


\subsection{Potential Applications} Subspace clustering can be a useful data mining tool in several applications where observations have too many registered values. 
For example, in bioinformatics, when doing gene expression analysis on gene expression data which contains the expression levels of thousands of genes. Briefly explained, an entry in such data describes the measurement of the level at which information from a gene is used to produce final gene products like protein or non-coding RNA. The expression levels can be measured at hundreds of different timestamps, different tissues, different persons or in different environmental conditions. We also know that genes have different functions under different circumstances. In other words, two genes may only behave coordinately in some environment and are otherwise dissimilar. That means trying to cluster genes across all dimensions would yield biologically inaccurate results. Since meaningful clusters of genes can only be found under a condition, subspace clustering seems to be a perfect solution. We hope by applying it to identify co-expressed genes that are functionally related and to use this information as a basis for further researches.
Subspace clustering of gene expression data shows that this approach, unlike classic clustering, can solve information extraction problems besides data grouping problems.   

An interesting benefit of applying subspace clustering on high dimensional financial data would be detecting anomalies in subspaces that otherwise could not be found when clustering the original feature space. That can be done by identifying points having an anomalous behavior compared to other data points. The reason this process could be so useful is its connection to financial fraud detection. Due to external influential factors from the outside-environment, irregularities could appear frequently in financial data in subspaces. Therefore, detecting an outlier does not necessarily indicate a fraud but it would maybe be enough to analyze the outlier and do start some investigations.

In medicine, subspace clustering of clinical data can be a potential application. For example, when discovering a new drug against some disease, data consisting of dozens of clinical parameters of patients that have taken the drug can be collected, such as lab values, diagnosis results of chronic diseases, medications, etc. The goal of subspace-clustering such high dimensional clinical data would be grouping patients based on their similar body’s reactions to the drug. Now besides groups of patients that have responded positively to the drug, there might be clusters of patients with similar irregular values in some clinical parameters, which may be caused by undesired side effects of the medication. In such a case, pharmaceutical scientists might find it necessary to develop a new medicine that fits those patients.

Today, companies collect great amounts of customers’ information, aiming to use the resulting knowledge out of analyzing this information in achieving more effective marketing. A possible application could be customer segmentation, where companies organize customers into segments based on similar characteristics like similar interests or dislikes. The company can then start offering a specific product or service based on what a segment’s members are interested in. Subspace clustering can be used here to create the wanted segments since traditional clustering methods would fail to handle high dimensional customers’ data correctly.
In general, subspace clustering methods can be used as a tool of business intelligence and be applied on any multi-dimensional customers’ data to extract information that helps business mangers in strategic planning or to make data-driven decisions.

\subsection{Alternative Methods} One can think of dimensionality reduction followed by full-space clustering as an alternative method to subspace clustering that helps avoid the effects of the curse of dimensionality when clustering high dimensional data. The assumption here is that high dimensional data contain a lot of redundant information. Therefore, it can be transformed from a high dimensional space into a lower dimensional space without a significant loss of information. We can differentiate between two approaches of dimensionality reduction, which are feature selection and feature projection. The first keeps a relevant subset of the original feature space based on information gain, measurements of accuracy or prediction errors. The second builds a new smaller set of features from functions of the original input features. The  most impactive drawback of dimensionality reduction techniques is being global, which means points will be clustered only in the one new found coordinate system. Therefore, the information of local clusters that can only be found in a relevant subset of features and overlapping clusters which contain points clustered differently in varying combinations of features would be lost. 

Due to this serious drawback, subspace clustering seems to be more promising. However it may be more computationally intensive. This can be explained by the two actual tasks happening in the process of subspace clustering, which are subspace searching and clustering.
 
Another straightforward alternative method would be to cluster the data objects in every possible subspace of the original feature space. It is to mention here that the number of possible subspaces is an exponential growth function of the number of all dimensions. Exactly there are $2^{d}-1$ different axis-aligned subspaces of a space with d dimensions. That means exploring each possible subspace would lead to a drastic runtime complexity of $O(2^{d})$. It would be expensive and impractical to consider all possible subspaces. Subspace clustering algorithms bypass this problem by using heuristic techniques to help filter out irrelevant subspaces and lower the computation time. 

\subsection{Classification of Subspace Clustering Methods} The task of clustering high dimensional data can be done by different approaches. We can distinguish between:
\begin{itemize}
\item Approaches searching for clusters only in axis-parallel subspaces: like subspace clustering, projected clustering, projection-based clustering and hybrid approaches. These approaches may differ in one or many aspects, such as allowing finding overlapping clusters, the used distance functions, used heuristics, the number of generated subspace clusters, ability to handle noise, etc, but they all work on grouping data objects based on similar values in a subset of features. Examples of such algorithms are CLIQUE \citep{10.1145/276305.276314}, SUBCLU \citep{subclu}, MAFIA \citep{DBLP:conf/sdm/NageshGC01} , PROCLUS \citep{10.1145/304181.304188}, FIRES \citep{fires} , P3C \citep{4053068} and ENCLUS \citep{10.1145/312129.312199}.

\item Approaches searching for clusters in axis-parallel subspaces and in special cases of axis-parallel and arbitrarily oriented subspaces: like pattern-based clustering. A common approach is bi-clutsering, which aims to cluster the rows and the columns simultaneously according to coherent patterns in the data matrix. Examples of such algorithms are FLOC \citep{10.1109/ICDE.2002.994771}, MaPle \citep{10.1109/ICDM.2003.1250928} and OP-Cluster \citep{10.1109/ICDM.2003.1250919}.

\item Approaches detecting clusters in axis-parallel and arbitrarily oriented subspaces: like correlation clustering. Such an approach is applied when features are complexly correlated so that data matrices do not contain any particular patterns, so algorithms in this field do not restrict the search to axis-parallel subspaces. As a result, this approach is computationally less efficient than only axis-parallel approaches because there are an infinite number of possible arbitrarily oriented subspaces in a space of d dimensions. Examples of such algorithms are ORCLUS \citep{10.1145/342009.335383}, EriC \citep{DBLP:conf/ssdbm/AchtertBKKZ07}, CASH \citep{DBLP:conf/sdm/AchtertBDKZ08} and COPAC \citep{DBLP:conf/sdm/AchtertBKKZ07}.
\end{itemize}
These various clustering approaches can be further classified based on shared characteristics, such as allowing detecting overlapping clusters and overlapping subspaces, used search methods or cluster model.

With regard to overlapping clusters, we can divide algorithms into two families, namely, algorithms that produce non-overlapping clusters by either assigning each data object to a unique cluster or marking it as a noise and algorithms that take into account the fact that data objects can belong to different clusters in varying subspaces and thereby work on finding all possible cluster in all subspaces of the feature space, which automatically lead to detect overlapping clusters. Examples of overlapping clusters algorithms are CLIQUE, SUBCLU, MAFIA, FIRES and ENCLUS. Examples of non-overlapping clusters algorithms are PROCLUS and PreDeCon \citep{DBLP:conf/icdm/BohmKKK04}.

When considering the used search methods, diverse approaches can be categorized into two groups depending on their methodology for navigation through the search space of possible subspaces. The first group includes methods that pursue a bottom-up approach, where methods of the second group pursue a top-down approach. 

Bottom-up approaches first begin with identifying all relevant 1-dimensional subspaces, then they combine those to build higher-dimensional subspaces and explore them. Detecting relevant 1-dimensional could be achieved by first applying a traditional clustering method to each 1-dimensional subspace. Secondly, keeping all 1-dimensional subspaces in which at least 1-dimensional cluster exist. 

This procedure can be explained by the monotonicity property used by many algorithms, which assumes that if a cluster exists in a subspace then this cluster also exists in all lower-dimensional projections of this subspace. To prune irrelevant dimensions, algorithms actually use the inversion of this assumption, which is as follows: if a dimension contains no clusters, then no higher-dimensional subspace of this dimension contains a cluster. This monotonicity assumption helps keep the bottom-up search efficient at the risk of producing lower quality results.

The combining step of subspaces is done as follows. Algorithms generate iteratively higher-dimensional subspaces by increasing the dimensionality by 1, so 1-dimensional subspaces are combined to build 2-dimensional ones, then 2-dimensional ones are combined to build 3-dimensional ones and so on, until no more potential higher-dimensional subspaces can be built. A special case of the bottom-up approach can be seen in FIRES \citep{fires} where 1-dimensional subspaces are combined to directly generate subspaces of maximum dimensionality.

Some algorithms do not generate high-dimensional subspaces by simply building combinations of lower-dimensional ones, instead they adopt a merging procedure to decide whether to combine subspaces or not. For example, by measuring the cardinality of the intersection of clusters in the related subspaces. Some examples of bottom-up algorithms are CLIQUE, SUBCLU, MAFIA, ENCLUS and P3C.
 
While on the contrary, top-down approaches start with detecting clusters in the full dimensional space and then iteratively learning the best subspace for each cluster by removing irrelevant dimensions. This is a cluster-based approach. An iterative instance-based approach starts in the full dimensional space with learning the preferential subspace for each data point, which is the subspace containing the best-fitting cluster for the point. Subsequently, points having similar preferential subspace are grouped together to form clusters. Most top-down algorithms compute a global discrete partitioning of the data objects, thus they do not allow overlapping clusters. Examples of such approaches are PROCLUS and COSA \citep{RePEc:bla:jorssb:v:66:y:2004:i:4:p:815-849}.

Approaches for clustering high-dimensional data can also be characterized by their cluster model into grid-based (also called cell-based) or density-based approaches.

The general idea of grid-based clustering approaches, also called cell-based approaches, is to partition a data space by an axis-aligned grid into a finite number of disjoint cells, then to determine dense cells and connect them to form clusters. Dense cells are grid cells containing a number of objects above a certain threshold and clusters are maximal sets of connected neighboring dense cells within a subspace. This procedure is done recursively during the navigation through the search space of possible subspaces.

Grid-based approaches can be further classified into static grid and adaptive grid approaches. Static grid approaches discretize the data space into equal sized cells all having the same width. CLIQUE is an example of such an algorithm. On the other hand, adaptive grid approaches arbitrarily partition a data space into variable sized cells, to allow maximizing the number of objects within particular cells. The algorithm MAFIA (Merging of Adaptive Finite IntervAls) is an example.

Grid-based approaches have a major drawback caused by the impact of the positioning of the grid. For example, some points on the edge of a cluster might be missed if they are not within dense cells or some points might be wrongly assigned to a cluster just because they exist in related dense cells, even though they are not actual cluster members. So the accuracy of the clustering could be negatively affected by the orientation and the shape of the clusters in relation to the position of the grid.

Density-based approaches are often more complex but capable of detecting clusters of any size and shape, thus more accurate. These approaches are based on the density-connected cluster paradigm introduced in DBSCAN \citep{10.5555/3001460.3001507}. By exploring the neighborhood of points and identifying core points in a subspace, these approaches detect regions of high density that are separated by regions of lower density. A cluster is defined as a maximal set of connected dense points. Examples of such algorithms are SUBCLU and FIRES.

Finally, subspace clustering approaches could be clustering-based regardless of whether they are grid-based or density-based, bottom-up or top-down. Such algorithms use global parameters to define properties of the expected clusters, like the number of clusters or the average dimensionality of these clusters. Examples of such algorithms are PROCLUS and P3C.

\subsection{Main Topic of Project} Now that we have basic knowledge about the various characteristics of approaches for clustering high-dimensional data and the different techniques and paradigms used by these approaches, we can turn our focus towards the main purposes of writing this modest thesis which is presenting an implementation of SUBCLU (density-connected Subspace Clustering) and FIRES (Filter Refinement Subspace clustering) in Python, two subspace clustering algorithms used to cluster high dimensional data. We will also compare both in many aspects and in different scenarios.

With these implementations we aim to help researchers by further studies in relation to SUBCLU and FIRES. Also, with some modifications, both implementations could be adapted to match the style of the pyclustering library (a Python, C++ data mining library) \citep{Novikov2019} and thus could be added to the list of implemented algorithms since SUBCLU and FIRES are yet not in the library.


\section{Overview of Related Work}\raggedbottom

To the best of our knowledge, the authors of SUBCLU and FIRES have not published any implementation of both the algorithms, which urged all researchers that have an interest in investigating the two algorithms to implement them by writing their own programs.
In addition, scientific researches that include SUBCLU or FIRES as part of some specific studies have only presented the end results of their study in relation to SUBCLU or FIRES, but no implementations were provided as this was not the main topic of these researches.

An example implementation of SUBCLU in Java \href{https://github.com/elki-project/elki/blob/master/elki-clustering/src/main/java/elki/clustering/subspace/SUBCLU.java}{SUBCLU.java} can be found in the official github repository of ELKI (Environment for Developing KDD-Applications Supported by Index-Structures), a general framework for data mining. Another implementation of SUBCLU and FIRES in Java should be found in the OpenSubspace framework, which is an Open Source Framework for Evaluation and Exploration of Subspace Clustering Algorithms in WEKA, but at the time of writing this thesis, the webpage of the project \href{http://dme.rwth-aachen.de/en/opensubspace}{http://dme.rwth-aachen.de/en/opensubspace} do not exist anymore.

There are some papers that have done comparative studies on both the algorithms. For example, in the FIRES \citep{fires} paper itself, after developing the algorithm, the authors tested its effectiveness against SUBCLU \citep{subclu} by applying both to synthetic data and then measuring the clustering quality, by first checking whether the hidden subspace clusters are discovered or not, and second by measuring how precise is the clustering in relation to noise. 
The authors then compared the scalability of SUBCLU and FIRES by measuring the runtime of both algorithms while increasing the number of points in the dataset, the data dimensionality and the dimensionality of a subspace cluster. These experiments have shown that FIRES is more accurate and more efficient. In addition, FIRES was more successfully applied on real-world high-dimensional gene expression data, unlike SUBCLU, which failed to detect some subspace clusters and caused memory overflow due to producing a high number of candidate subspaces.

Another evaluation of both algorithms can be found in the paper titled “Evaluating Clustering in Subspace Projections of High Dimensional Data” \citep{10.14778/1687627.1687770}. This publication aimed to present a fair and thorough evaluation and comparison of different developed subspace clustering paradigms by applying evaluation criteria thoughtfully so that all paradigms are tested under impartial experimental conditions. 

The authors did object-based and object- and subspace-based external evaluation measures with respect to the ground truth provided by synthetic data. Object-based measures included measuring the entropy (the purity of the found clusters with respect to the actual hidden subspace clusters in the data), F1 (to evaluate how well the hidden clusters are represented by the found clusters), accuracy (to evaluate how good the ground truth is generalized by the identified clusters).

Whereas object- and subspace-based evaluation measures included evaluation performance parameters in relation to the relevant dimensions of the subspace clusters. For these measurements, each data object in a D-dimensional database was partitioned into D different subobjects, also called micro-objects, so that subspace clusters can be represented as subsets of micro-objects. The first included an evaluation approach called RNIA (relative non-intersecting area) that measures how well the subobjects of the ground truth are covered by the found subobjects. The second evaluation approach is called CE (clustering error) which fulfills the same quality criteria as RNIA, but penalizes clusters which split up into several smaller ones.

The performance evaluation results show that SUBCLU detects much more subspace clusters than FIRES does and therefore, SUBCLU has a better coverage score but worse scores with respect to entropy, RNIA and CE. SUBCLU has also  achieved better F1 and accuracy scores. However, evaluating the scalability of SUBCLU and FIRES with respect to data dimensionality and database size, shows that SUBCLU could not compete against FIRES, as the runtime of SUBCLU could be extremely high, up to several days, if the data dimensionality or size above some certain numbers. The authors used their open source framework \textit{OpenSubspace} for the analysis. All implementations, datasets and evaluation measures should actually be available on the website  \href{http://dme.rwth-aachen.de/OpenSubspace/evaluation}{http://dme.rwth-aachen.de/OpenSubspace/evaluation} as mentioned in the article.

Another comparison of FIRES and SUBCLU using the novel evaluation measure E4SC can be seen in \citep{10.1145/2063576.2063774}. The E4SC measure has all the characteristics of a complete external subspace evaluation measure since it fulfills all general quality criteria for subspace clustering measures.


\section{Fundamentals}\raggedbottom

Although there are various approaches that address the problem of subspace clustering, the fundamental terms are globally defined. Three of these terms are explained as follows:

\textbf{Definition 1 (Subspace)} Let $DB$ be a data set and $A$ be the set of all attributes spanning the feature space of DB. Any subset $S \subseteq A$ of the original feature space is a subspace. 

\textbf{Definition 2 (Subspace Cluster)} Let $n$ objects be in $DB$. A group of $m \leq n $ objects that are similar in all dimensions of a subspace $S$ is called a subspace cluster. 

\textbf{Definition 3 (Dimensionality of Subspace Cluster)} The number of dimensions $(|S|)$ forming the subspace in which the cluster exists.

\textbf{Definition 4 (Overlapping Subspace Clusters)} Let $C_{i}$ be a cluster in subspace $S_{i}$ and $C_{j}$ a cluster in subspace $S_{j}$. We call $C_{i}$ and $C_{j}$ overlapped, if ($C_{i} \neq C_{j}$), $(C_{i} \cap C_{j}) \neq \emptyset$ and $(S_{i} \cap S_{j}) = \emptyset$.

\textbf{Definition 5 (Overlapping Subspaces)} Let $C_{i}$ be a cluster in subspace $S_{i}$ and $C_{j}$ a cluster in subspace $S_{j}$. We call $S_{i}$ and $S_{j}$ overlapped, if ($S_{i} \neq S_{j}$), $(S_{i} \cap S_{j}) \neq \emptyset$ and $(C_{i} \cap C_{j}) = \emptyset$.

\textbf{Definition 6 (Simultaneously Overlapping Clusters and Subspaces)} Let $C_{i}$ be a cluster in subspace $S_{i}$ and $C_{j} \neq C_{i}$ a cluster in subspace $S_{j} \neq S_{i}$. We call $C_{i}$, $C_{j}$ and $S_{i}$, $S_{j}$ overlapped, if $(C_{i} \cap C_{j}) \neq \emptyset$ and $(S_{i} \cap S_{j}) \neq \emptyset$.

Any other not general technical concepts that reflect the understanding of authors will be defined in later sections.

\section{SUBCLU and FIRES}\raggedbottom

\subsection{SUBCLU}
SUBCLU \citep{subclu} is the first subspace clustering method for high-dimensional data that has a density-based cluster model of DBSCAN. It searches for clusters only in axis-parallel subspaces and it is capable of detecting both overlapping clusters and overlapping subspaces. It can find subspace clusters of arbitrary size, shape and dimensionality. It is deterministic, robust to noise, generates an arbitrary number of subspace clusters, independent with respect to the order of features and to the order of objects. 

The algorithm can be represented by two repeatedly done major steps, generating candidate subspaces and clustering data objects in these subspaces.

Since SUBCLU uses DBSCAN \citep{10.5555/3001460.3001507} as a clustering method, subspace clusters detected by SUBCLU are maximal sets of density-connected points whereby density connectivity is defined based on core points that have at least \textit{minPts} points in their $\varepsilon$-neighborhood. SUBCLU takes $minPts$ and $\varepsilon$ as parameters and employs them in all clustering calls and therefore some subspace clusters of certain local densities may remain undetected due to the globally used density threshold $\varepsilon$.

Before going ahead and explaining the workflow of SUBCLU, it is important to understand the monotonicity property used to reduce the search space by pruning subspaces that cannot contain any clusters. The monotonicity assumption states that if a cluster $C$ exists in a subspace $S$, then it should also exist in all lower-dimensional subspaces $T \subseteq S$. Therefore, we can remove all subspaces $T$ that contain no clusters since no superspace $T \subset S$ contains a cluster. This monotonicity assumption can be explained by the monotonicity of density-connected sets, which is proven in the publication of SUBCLU \citep{subclu}.

The algorithm has a bottom-up search approach. It starts with identifying all important 1-dimensional subspaces $S_{1}$ by applying DBSCAN to each 1-dimensional subspace individually and then checking whether any 1-dimensional clusters $C_{1}$ (also called base-clusters) were found or not. All 1-dimensional subspaces that contain no clusters are considered not relevant and are not used to build candidate subspaces of higher dimensionalities. 

SUBCLU increases the dimensionality of searched subspaces and subspace clusters by 1 for each iteration. That means after computing $S_{1}$ and $C_{1}$, if any base-clusters were found $C_{1} \neq \emptyset$, then SUBCLU generates all relevant $(k + 1)$-dimensional candidate subspaces $CandS_{k+1}$ and checks whether detected clusters are conserved in these candidates subspaces or not, whereby at the beginning $k = 1$ for $k \in \{1,....,d\}$ and $d$ being the dimensionality of the database.  

Creating the $(k + 1)$-dimensional candidate subspaces is done by joining every two unequal $(k)$-dimensional candidates together if they have $(k - 1)$ attributes in common. This is meant for 3 dimensions and greater ($k > 1$). For building the 2-dimensional candidate subspaces, SUBCLU simply considers all combinations of every two attributes in $S_{1}$ without replacement.

According to the above-mentioned monotonicity assumption, SUBCLU prunes each generated $(k + 1)$-dimensional candidate subspace $cand \in CandS_{k+1}$ if at least one $(k)$-dimensional subspace $T \subseteq cand$ contains no cluster. After pruning the set $CandS_{k+1}$, SUBCLU ends the iteration by applying DBSCAN to each $cand \in CandS_{k+1}$ and computing $S_{k+1}$ and $C_{k+1}$, the set of $(k + 1)$-dimensional subspaces containing clusters and the set of found clusters in these $(k + 1)$-dimensional subspaces respectively.

To minimize the the number of range queries necessary during the runs of DBSCAN in $cand$ and thus the overall cost, DBSCAN is not applied on all found clusters in all subspaces of $cand$ but only on the objects of clusters in a so-called $bestSubspace \subset cand$ which is the subspace with the minimum number of objects in the clusters:
\begin{equation}
bestSubspace = \min _{s\in S_{k}\wedge s\subset cand}\sum _{C_{i}\in C^{s}}|C_{i}|.
\end{equation}
$C^{s}$ denotes the set of all detected clusters in subspace $s$.

The process of generating and pruning candidate subspaces and clustering data objects is done recursively until no new clusters in higher-subspaces can be found.

After reading the publication SUBCLU \citep{subclu} we would like to highlight some unclarities in the paper:

\begin{itemize}
	\item It is not clear which clusters are to be returned. Since the authors did not provide any redundancy removal method or suggested one, all found subspaces clusters of dimensionality 1 and up to maximum dimensionality are returned. Therefore, SUBCLU produces a large number of subspace clusters. An example can be seen in the article \citep{10.14778/1687627.1687770} Figure 13, when clustering the Vowel dataset with 990 records, SUBCLU generated 709--10881 clusters.
	\item SUBCLU is a greedy algorithm, because of the locally cost-optimal choice of $bestSubspace$, but the authors did not clarify how to handle the case of having many $bestSubspace$. Whether we consider them all or whether we choose wisely one of them is not discussed. 
\end{itemize}
We also would like to mention some remarks about SUBCLU:
\begin{itemize}
	\item The algorithm fails to drop irrelevant dimensions that contain no clusters because DBSCAN assigns almost all objects to one big cluster in these dimensions and therefore, they are considered relevant dimensions as they have at least one cluster and they are used for building $(k + 1)$-dimensional candidate subspaces. As a result, all dimensions of the feature space are involved, thus SUBCLU can not avoid a complete enumeration of all subspaces and it will end up in the worst case algorithm's complexity $O(2^{d})$ since there are $2^{d}-1$ different axis-parallel subspaces for a databse of $d$-dimensional feature space.
 	\item The 2-dimensional candidate subspaces in the set $CandS_{2}$ need not be checked for pruning since every 2-dimensional candidate subspace is built by joining two 1-dimensional subspaces from $S_{1}$ each containing at least one cluster. 
\end{itemize}


\subsection{FIRES}
FIRES (FIlter REfinement Subspace clustering) \citep{fires} is a subspace clustering framework with no restrictions to the underlying clustering method used to cluster data objects in subspaces.

FIRES was classified as a subspace clustering approach by its authors, but to be precise, FIRES, in addition, is a hybrid approach for clustering high-dimensional data, due to its unique way of searching for subspace clusters. Where unlike other subspace clustering approaches that work on detecting all hidden clusters in all subspaces, FIRES computes subspace clusters of maximal-dimensionalities by making use of the monotonicity assumption mentioned above and skipping on all subspace clusters within projections of these maximal-dimensional subspace clusters. This strategy helps FIRES to avoid a complete enumeration of all subspaces, it keeps FIRES efficient and the number of produced cluster low.

In comparison to SUBCLU, applying FIRES on the Vowel dataset \citep{10.14778/1687627.1687770} yielded only 24--32 subspace clusters.

FIRES searches for clusters only in axis-aligned subspaces and it is capable of detecting both overlapping clusters and overlapping subspaces. It can find subspace clusters of arbitrary size, shape and maximal-dimensionality. It is deterministic, robust to noise, generates an arbitrary number of subspace clusters and independent with respect to the order of objects.

The algorithm has three major steps:
\begin{itemize}
	\item Preclustering.
	\item Generation of Subspace Cluster Approximations.
	\item Postprocessing of Subspace Clusters.
\end{itemize}
But unlike SUBCLU, FIRES is not recursive, it goes through the three steps once and returns a resulting clustering.

To get a better overview and better understanding, we wrote pseudocode (Algorithm~\ref{alg:fires}) to describe all steps in the algorithm since the original publication of FIRES does not include one.

\RestyleAlgo{ruled}
\IncMargin{1em}
\begin{algorithm}[H]
	\DontPrintSemicolon
	FIRES(database $D$, int $ \mu $, int $ k $, int $ minClu $, object clusteringMethod):\;
	$C^{1} = \emptyset$	// set of all 1-dimensional subspace clusters\;
	$MSC_{k} = \{\}$ // k-most-similar-clusters\;
	$BMC = \{\}$ // best-merge-candidates\;
	$best$-$merge$-$clusters = \{\}$\;
	$subspace$-$cluster$-$approximations = \{\}$\;
	$clustering = \{\}$ // end result\;
	\textcolor{teal}{/* STEP 1 Generate all base-clusters */\;}
	\ForEach{dimension $a_{i} \in A$}{
		$C^{a_{i}}$ = clusteringMethod$(D, a_{i})$ // set of all clusters in subspace $a_{i}$\;
		\If{$C^{a_{i}} \neq \emptyset$}{
			$C^{1} = C^{1} \cup C^{a_{i}}$\;
		}
	}
	\textcolor{teal}{/* STEP 2 Prune irrelevant base clusters */\;}
	$s_{avg} = \frac{\sum _{\forall c_{s}\in C^{1}}|c_{s}|}{|C^{1}|}$ //average size of all base-clusters\;
	\ForEach{base-cluster $c_{s} \in C^{1}$}{
		\If{$|c_{s}| < \frac{s_{avg}}{4}$}{
			$C^{1} = C^{1} \setminus c_{s}$\;
		}
	}
	\textcolor{teal}{/* STEP 3 Check base clusters for splits */\;}
	$s_{avg} = \frac{\sum _{\forall c_{s}\in C^{1}}|c_{s}|}{|C^{1}|}$\;
	repeat {\;  
		\ForEach{base-cluster $c_{s} \in C^{1}$}{
			$MSC(c_{s}) = c_{t}$ for $c_{t} \in C^{1} \wedge c_{t} \neq c_{s} \wedge (|c_{t} \cap c_{s}|) \geq (|c_{p} \cap c_{s}|) \forall c_{p} \in C^{1}$ // most similar cluster\;
			\If{$|c_{s} \cap c_{t}| \geq \frac{2s_{avg}}{3} \wedge |c_{s} \setminus c_{t}| \geq \frac{2s_{avg}}{3}$}{
				$C^{1} = C^{1} \setminus c_{s}$\;
				$C^{1} = C^{1} \cup (c_{s} \cap c_{t}) \cup (c_{s} \setminus c_{t})$\;
			}
		} until {\;
			$\forall c_{s} \in C^{1}, |c_{s} \cap MSC(c_{s})| < \frac{2s_{avg}}{3} \lor |c_{s} \setminus MSC(c_{s})| < \frac{2s_{avg}}{3}$\;
		}
	}
	\textcolor{teal}{/* STEP 4 Compute k-most-similar-clusters */\;}
	\ForEach{base-cluster $c \in C^{1}$}{
		$MSC_{k}[c] = \{c_{1},\ldots,c_{k} | c_{1},\ldots,c_{k} \neq c \wedge c_{1},\ldots,c_{k} \subset C^{1} \wedge \forall c_{i} \in \{c_{1},\ldots,c_{k}\}, \forall c_{q} \in C^{1} \setminus (c_{1},\ldots,c_{k}): |c_{i} \cap c| > |c_{q} \cap c|$\}\;
	}
	\caption{The FIRES Algorithm}\label{alg:fires}
\end{algorithm}
\IncMargin{1em}
\begin{algorithm}
	\DontPrintSemicolon
	\textcolor{teal}{/* STEP 5 Compute best merge candidates */\;}
	\ForEach{base-cluster $c \in C^{1}$}{
		\ForEach{base-cluster $x \in C^{1} \setminus c$}{
			\If{$|MSC_{k}[c] \cap MSC_{k}[x]| \geq \mu$}{
				$BMC[c] = BMC[c] \cup x$\;
			}
			
		}
	}
	\textcolor{teal}{/* STEP 6 Compute best merge clusters */\;}
	\ForEach{base-cluster $c \in C^{1}$}{
		\If{$|BMC[c]| \geq minClu$}{
			$best$-$merge$-$clusters = best$-$merge$-$clusters \cup c$\;
		}
		
	}
	\textcolor{teal}{/* STEP 7 Generate subspace cluster approximations */\;}
	\ForEach {base-clusters $c_{A}, c_{B} \in best$-$merge$-$clusters,c_{A} \neq c_{B}$}{
		\If{$c_{A} \in BMC[c_{B}] \wedge c_{B} \in BMC[c_{A}]$}{
			$subspace$-$cluster$-$approximations =$ $subspace$-$cluster$-$approximations \cup (BMC[c_{A}] \cup BMC[c_{B}])$\;
		}
		
	}
	\textcolor{teal}{/* STEP 8 Prune subspace cluster approximations */\;}
	\ForEach {approximation $C \in subspace$-$cluster$-$approximations$}{
		$clean =$ false\;
		\While{$\neg clean$ }{
			$clean =$ true\;
			$score(C) =|\bigcap c_{i} \in C|$ dim($C$)//dim is the dimensionality of a cluster\;
			\ForEach{$c \in C$}{
				\If{$score(C\setminus c) > score(C) \wedge score(C\setminus c) \geq score(C\setminus c_{p}), \forall c_{p} \in C: c_{p} \neq c$}{
					$C = C \setminus c$\;
					$clean =$ false\;
					break\;
				}
			}
		}	
	}
	\textcolor{teal}{/* STEP 9 Refine subspace cluster approximations */\;}
	\ForEach {approximation $C \in subspace$-$cluster$-$approximations$}{
		$union(C) = \bigcup c_{i} \in C$ // union of all base-clusters in C\;
		$subspace(C) = \bigcup$ getSubspace$(c_{i}) \in C$ // corresponding dimensions\;
		$C_{sub} =$ clusteringMethod$(union(C), subspace(C))$\;
		$clustering = clustering \cup C_{sub}$\;
	}
\end{algorithm}

The preclustering starts with identifying all 1-dimensional clusters, called base-clusters, (step 1 Algorithm~\ref{alg:fires}). Since FIRES is a generic framework, this could be done by applying any clustering method like k-means, DBSCAN, CLIQUE or others. FIRES then removes all base-clusters with cardinality less than 25\% of the average size of all base-clusters (step 2 Algorithm~\ref{alg:fires}) as FIRES do not consider these small base-clusters to be parts of clusters in subspaces of higher-dimensionality.

After removing unpromising base-clusters, FIRES starts the process of generating subspace cluster approximations. This major step represents the actual clustering idea behind the FIRES algorithm which is grouping similar base-clusters lying in different dimensions to generate approximations of subspace cluster of form $approximation_{i} = (C_{i}, S_{i})$, for $C_{i} = \{c_{1},\ldots,c_{k}\}$ being a set of similar base-clusters and $S_{i} = \{s_{1},\ldots,s_{k}\}$ the corresponding dimensions in which the base-clusters exist. But before searching for similar base-clusters, FIRES first splits untruly similar base-clusters that have objects from overlapped clusters and will split in higher-dimensional subspaces (step 3 Algorithm~\ref{alg:fires}) in order to avoid merging them with perfectly matched clusters. This is done by checking whether the intersection between a base-cluster and its most-similar-cluster together with the difference are greater than two-thirds of the average size of all base-clusters. One base-cluster could split multiple times, each into intersection and difference.

Now that all base-clusters are ready to be merged, Fires searches for each base-cluster its $k$-most-similar-clusters that share objects with it the most (step 4 Algorithm~\ref{alg:fires}). Next FIRES computes for each base-cluster its best-merge-candidates set that contains other base-clusters having at least $\mu$-most-similar-clusters in common(step 5 Algorithm~\ref{alg:fires}) and finally FIRES computes the set of best-merge-clusters that contains base-clusters having at least $minClu$ best-merge-candidates (step 6 Algorithm~\ref{alg:fires}).To generate the subspace cluster approximations, FIRES merges every two best-merge-clusters with their best-merge-candidates together if both are best-merge-candidates of each other (step 7 Algorithm~\ref{alg:fires}).

Since subspace cluster approximations are created based on information of the base-clusters, they can differ from clusters that might be found when applying a traditional clustering algorithm to each subspace directly, due to the fact that clusters' members and size usually do not remain the same while increasing dimensionality. Therefore, the algorithm provides an optional postprocessing step that improves the quality of the created approximations and refines them to their final form.

This step involves first removing base-clusters from approximations if they decrease the number of objects shared by all base-clusters in the approximations(i.e. not that similar to other base-clusters in the set) (step 8 Algorithm~\ref{alg:fires}) and second applying the same clustering method used in (step 1 Algorithm~\ref{alg:fires}) to identify base-clusters on the union of base-clusters within the approximations (step 9 Algorithm~\ref{alg:fires}). 

FIRES expects clusters to be differently dense in relation to the dimensionality of the subspaces in which they exist. Therefore, it uses an adaptive density threshold and adjusts it with respect to the subspace dimensionality. For example, if DBSCAN was chosen as a clustering method in (step 1 Algorithm~\ref{alg:fires}) then the authors suggest to redefine $\varepsilon$ for each subspace cluster in (step 9 Algorithm~\ref{alg:fires}) as $\varepsilon =  \frac{\varepsilon_{1}n}{\sqrt[d]{n}}$ where $d$ is the subspace cluster dimensioality, $\varepsilon_{1}$ is the density threshold used in the preclustering (step 1 Algorithm~\ref{alg:fires}) and $n$ is the number on points in the dataset.

In order to better understand the clustering model of FIRES, in particularly how subspace cluster approximations are generated, we will demonstrate an example using directed graphs. This example is based on information generated by applying FIRES on a synthetic dataset with 1000 objects and 8 dimensions. The dataset includes 7 subspace clusters of various dimensionalities between 2--5.

Using DBSCSN with parameters $\varepsilon = 0.2$ and $minPts = 6$ as the clustering method, FIRES has detected 32 base-clusters indexed from 0--31. The vertices of the graph represent these base-clusters and the edges represent the $k$-most-similar-clusters.
\begin{figure}[h]
	\centering
	\includegraphics[clip, trim=1cm 4.5cm 1cm 3.2cm, width=1.0\textwidth]{bilder/kEqual2}
	\caption{Example of best merge clusters and their best merge candidates. $k = 2, \mu = 2$ and $minClu = 2.$}
	\label{fig:kEqual2}
\end{figure}

In Figure~\ref{fig:kEqual2} every node has two outgoing edges since we chose $k = 2$. Blue clusters 1, 14 and 18 share their 2-most-similar-clusters, namely clusters 8 and 19, and therefore they are best-merge-candidates of each other, since they have at least $\mu$-similar-clusters in common. This also applies to the pink clusters 0, 2, 19 and 27 as they all share at least 2-most-similar-clusters 15 and 23.

Colorful nodes form the set of best-merge-clusters because each one has at least $minClu$-best-merge-candidates. On the other hand, clusters 26 and 30 are best-merge-candidates of each other since they fulfill the condition of sharing $\mu$-similar-clusters but they are not part of the best-merge-clusters since each has only one best-merge-candidate. The same goes for clusters 8, 31 with the green edges, clusters 10, 17 with the orange edges and clusters 20, 28 with the red edges.

To build the subspace cluster approximations, FIRES checks every possible pair of the best-merge-clusters (i.e. colorful nodes) if they are best-merge-candidates of each other (i.e. have the same color) and if this is the case, FIRES groups both best-merge-clusters with their best-merge-candidates together. So, in this example, we have two subspace cluster approximations. First one consists of clusters (1, 14, 18) and the second on of clusters (0, 2, 19, 27). For each approximation, the points within the clusters are merged and reclustered in the corresponding subspaces. For example, if the base-clusters 1, 14, 18 exist in dimensions $a, b ,c$ respectively, where $a\neq b\neq c, a\neq c$, then their merged points are clustered in subspace $(a,b,c)$ to generate a subspace cluster of dimensionality 3.

As we mentioned, there are 7 subspace clusters within the dataset, which means there should be 7 groups of different colors and not only 2 as in Figure~\ref{fig:kEqual2}. So we reapplied FIRES to the dataset with parameter $k = 3$. The results are shown in Figure~\ref{fig:kEqual3} below.
\begin{figure}[h]
	\centering
	\includegraphics[clip, trim=1cm 4cm 1cm 5.2cm, width=1.0\textwidth]{bilder/kEqual3}
	\caption{Example of best merge clusters and their best merge candidates. $k = 3, \mu = 2$ and $minClu = 2.$}
	\label{fig:kEqual3}
\end{figure}

Every node has 3 outgoing edges representing its 3-most-similar-clusters. Now more clusters have at least $minClu$-best-merge-candidates and therefore, more clusters are in the set of best-merge-clusters. Even though cluster 21(on the right side) is not a best-merge-cluster since it has only cluster 29 as its best-merge-candidate, it still appears in the subspace cluster approximations and will be merged with the other cyan clusters, because it is a best-merge-candidate of cluster 29. This also applies to clusters 4, 12, 30 and 31, except they will be merged with other clusters than the cyan ones, since they are best-merge-candidates of clusters 2, 1, 26 and 8 respectively. We know this result is incorrect since there are nine pink colored clusters that should form a subspace cluster of dimensionality nine, which is impossible in a dataset with eight dimensions. This leads to the fact that at least two clusters of the nine pink clusters are in the same dimension. Here clusters 0 and 2 are in the same dimension, as well as clusters 5 and 6. Merging these clusters together is incorrect, since in hard clustering, no clusters in the same subspace can be similar.

Finally, since the dataset is artificially generated and the hidden subspace clusters are known, we would like to show the best result we could achieve and the expected subspace cluster approximations in Figure~\ref{fig:bestResult} and Figure~\ref{fig:groundTruth} respectively.

\begin{figure}[H]
	\centering
	\begin{minipage}{0.5\textwidth}
		\centering
		\includegraphics[width=0.9\textwidth]{bilder/bestResult}
		\caption{$k = 4, \mu = 3$ and $minClu = 1.$}
		\label{fig:bestResult}
	\end{minipage}\hfill
	\begin{minipage}{0.5\textwidth}
		\centering
		\includegraphics[width=0.9\textwidth]{bilder/groundTruth}
		\caption{Desired result.}
		\label{fig:groundTruth}
	\end{minipage}
\end{figure}  

In Figure~\ref{fig:bestResult} we kept only the edges needed for explanation and removed the rest to keep it visible.

First thing to notice is the pink-colored base-clusters. FIRES keeps on grouping them together as a mergerable-set although the dataset does not include this information. After examining these base-clusters, we found out that each consists of the remaining objects in the dataset that are not in any of the artificially-generated subspace clusters.

FIRES failed to detect the orange cluster in the full subspace dimensionality 4. As we can see in Figure~\ref{fig:bestResult}, cluster 13 is not colored orange because it does not have at least 3-most-similar-clusters in common with clusters 10, 17, 22. In addition, FIRES failed to detect the yellow subspace cluster.

As we did for SUBCLU, we would like to mention some unclarities in the paper of FIRES \citep{fires}:

\begin{itemize}
	\item The authors made it clear that a base-cluster may be split multiple times (step 3 Algorithm~\ref{alg:fires}). However, it is unclear whether we split clusters that satisfy the conditions directly or whether we mark them for split and do it later after iterating over all base-clusters. Also, Whether we recompute $s_{avg}$ after each split or if it is recomputed at all is not discussed.
	\item When computing the $k$-most-similar-clusters (step 4 Algorithm~\ref{alg:fires}). It is unclear which $k$-most-similar-clusters to consider in case there are more than $k$ similar ones. We think in such a case the right approach is to choose the $k$ ones having the least difference, since they are more similar than others.  
\end{itemize}
Also some remarks about FIRES:
\begin{itemize}
	\item The resulting clustering of FIRES is heavily dependent on the input parameters. As we saw in the example in Figure~\ref{fig:kEqual2} and Figure~\ref{fig:kEqual3}, changing one hyperparameter $k, \mu$ or $minClu$ leads to significantly different results. Besides these three hyperparameters, we also have the input parameters of the used clustering method. For example, if DBSCAN is chosen, there will be five parameters: $k, \mu, minClu, \varepsilon$ and $minPts$ that can affect the output. So tuning these hyperparameters to find the optimal setting for each dataset is problematic.
	\item Merging clusters that are detected in the same dimension is not avoidable. As we saw in Figure~\ref{fig:kEqual3}, clusters 0, 2 and 5, 6 are in the same subspace cluster approximation. None of these clusters will be removed in (step 8 Algorithm~\ref{alg:fires}), because grouping these clusters together decreases the quality of the approximation $score(C)$ down to zero, since it is calculated based on the intersection between all base-clusters within the approximation. $score(C\setminus c)$ will also be equal zero because no matter which base-cluster we exclude, there will always be two clusters with no intersection between them. Thus, the condition is never satisfied. 
	\item Imprecise and impure base-clusters that have some noise points or  points from other classes besides the true ones are penalized even though it is normal for clusters to have some incorrectly assigned points. This happens because FIRES uses the intersection between clusters as a similarity measure. An Example is cluster 13 in Figure~\ref{fig:bestResult}. It has a greater intersection with clusters 4 and 23 than with clusters 10 and 17. Therefore, it fails to have at least $\mu$-most-similar-clusters in common with the other orange clusters and it will not be merged with them. Due to this behavior, some clusters are not detected in the desired subspaces but in lower-dimensional projections of them.
	\item True base-clusters might be merged with big base-clusters consisting of the remaining objects in the dataset that are not in any of the artificially-generated clusters. For example, clusters 2 and 6 in Figure~\ref{fig:kEqual3}. Like other big base-clusters, both 2 and 6 have clusters 15 and 23 in their $k$-most-similar-clusters set and therefore, they will be merged with them. This happens because the similarity function based on the intersection between clusters prefers big clusters that have a lot of points.  
	\item Since there are no restrictions about the choice of the $k$-most-similar-clusters to consider in case there are more than $k$ ones, either the first $k$ ones or the last $k$ ones are chosen. In both cases, shuffling the attributes of a dataset can lead to different clusterings.  Therefore, FIRES is not independent with respect to the order of attributes.
	\item The suggested redefinition of $\varepsilon =  \frac{\varepsilon_{1}n}{\sqrt[d]{n}}$ for the postprocessing step in the case of choosing DBSCAN as a clustering method is questionable. Because the recomputed value could be very high and therefore, subspace clusters could merge. For example, if we have a dataset with 1000 observations, a subspace cluster of dimensionality 5 and we choose $\varepsilon_{1} = 0.5$ then the new $\varepsilon$ will be $\varepsilon = \frac{0.5 \cdot 1000}{\sqrt[5]{1000}} \approx 125$ which is too high compared to $\varepsilon_{1}$.
	\item After pruning all subspace cluster approximations (step 8 Algorithm~\ref{alg:fires}) we need to search for duplicates and strict subsets and remove them. Because according to the monotonicity assumption, strict subsets represent the clusters in lower-dimensional projections of the subspaces and thus can be removed as redundancy.
\end{itemize}

\section{Examples}\raggedbottom

In general, when clustering by FIRES we struggled to find the right combination of hyperparameters for each dataset and that applies to the two examples illustrated below. We have also noticed that the more subspace clusters with various dimensionalities there are, the harder it is to tune the parameters in a way that allows all clusters to be detected.

Since DBSCAN was chosen as a clustering method for FIRES in the following examples, we started the process of tuning the hyperparameters by finding proper $\varepsilon$ and $minPts$ values, because they are the key for detecting base-clusters that form the basis for all other steps in the algorithm. After fixing $\varepsilon$ and $minPts$, we tried to find a combination of $k, \mu$ and $minClu$ that leads to good balanced evaluation scores while taking into account the most successful values of these parameters that are mentioned in the paper \citep{fires}.

For SUBCLU, the density threshold is globally defined and therefore $\varepsilon$ should not be chosen too low in order to detect clusters in high-dimensional subspaces, the thing that may keep some low-dimensional subspace clusters undetected.

We applied SUBCLU and FIRES to synthetic datasets\footnote{\href{https://github.com/david-c-hunn/edu.uwb.opensubspace/tree/master/edu.uwb.opensubspace/Databases}{Link to Github.}} with unequal subspace-cluster density and dimensionality. 

The first dataset is D05 with 1595 observations and 5 dimensions. It contains 10 subspace clusters of dimensionalities 3 and 4. It also contains simultaneously overlapped clusters and subspaces. To visualize the multidimensional data and see if we can identify 10 subspace clusters, we applied a dimensionality reduction method, called PCA (principal component analysis) on D05 using three principal components and transformed the feature space to a new reduced coordinate system. Subsequently, we generated a 3D scatter plot that can be seen in Figure~\ref{fig:D05}:
\begin{figure}[H]
	\centering
	\centering
	\includegraphics[clip, trim=1cm 3.5cm 1cm 4.5cm, width=1.0\textwidth]{bilder/D05}
	\caption{3D scatter plot of D05.}
	\label{fig:D05}
\end{figure}
As we can see, there are only 8 different colors that represent 8 different subspace clusters. In order to see the remaining 2, we have to slightly move the interactive 3D plot to get another perspective:
\begin{figure}[H]
	\centering
	\includegraphics[clip, trim=1cm 2.7cm 1cm 3cm, width=1.0\textwidth]{bilder/D05AnotherPerspective}
	\caption{Another perspective of the 3D scatter plot of D05.}
	\label{fig:D05AnotherPerspective}
\end{figure}
Now we can see a red and a gray subspace cluster which overlap with the blue and orange clusters respectively.

After applying SUBCLU and FIRES with DBSCAN as a clustering method, we examined the clustering results of both algorithms on D05 and found the following:
\begin{table}[H]
	\centering
	\begin{tabular}{|c|c|c|c|c|}
		\hline
		\small Subspace&\small Cluster&\small SUBCLU&\small FIRES&\small Color\\
		\hline
		\hline
		$\{0,1,4\}$&\small $\{0,\dots,302\}$&\checkmark&\ding{53}&\textcolor{blue}{\ding{108}}\\
		\hline
		\hline
		\multirow{3}{3.8em}{$\{0,1,3,4\}$}&\small $\{0,\dots,156\}$&\checkmark&\ding{53}&\textcolor{red}{\ding{108}}\\
		&\small $\{1008,\dots,1158\}$&\checkmark&\checkmark&\textcolor{Salmon}{\ding{108}}\\
		&\small $\{1159,\dots,1309\}$&\checkmark&\checkmark&\textcolor{black}{\ding{108}}\\
		\hline
		\hline
		$\{1,2,3\}$&\small $\{303,\dots,604\}$&\checkmark&\ding{53}&\textcolor{orange}{\ding{108}}\\
		\hline
		\hline
		$\{1,2,3,4\}$&\small $\{303,\dots,459\}$&\checkmark&\checkmark&\textcolor{gray}{\ding{108}}\\
		\hline
		\hline
		\multirow{2}{3.8em}{$\{0,2,3\}$}&\small $\{605,\dots,755\}$&\checkmark&\ding{53}&\textcolor{cyan}{\ding{108}}\\
		&\small $\{1310,\dots,1460\}$&\checkmark&\checkmark&\textcolor{YellowGreen}{\ding{108}}\\
		\hline
		\hline
		$\{1,2,4\}$&\small $\{605,\dots,655,756,\dots,854\}$&\checkmark&\ding{53}&\textcolor{magenta}{\ding{108}}\\
		\hline
		\hline
		$\{0,2,3,4\}$&\small $\{855,\dots,1007\}$&\checkmark&\checkmark&\textcolor{yellow}{\ding{108}}\\
		\hline
	\end{tabular}
	\caption{Clustering results on dataset D05.}
	\label{tab:D05Report}
\end{table}

Another dataset, called D10, consists of 1595 observations and 10 attributes. There are 10 subspace clusters of dimensionalities between 5--8. As for D05, simultaneously overlapped clusters and subspaces do exist. The clustering results are illustrated below: 
\begin{table}[H]
	\centering
	\begin{tabular}{|c|c|c|c|}
		\hline
		\small Subspace&\small Cluster&\small SUBCLU&\small FIRES\\
		\hline
		\hline
		$\{1,2,3,5,9\}$&\small $\{0,\dots,300\}$&\checkmark&\ding{53}\\
		\hline
		\hline
		{$\{0,1,2,3,5,6,7,9\}$}&\small $\{0,\dots,150\}$&\checkmark&\ding{53}\\
		\hline
		\hline
		$\{0,2,4,6,9\}$&\small $\{301,\dots,602\}$&\checkmark&\ding{53}\\
		\hline
		\hline
		$\{0,1,2,3,4,5,6,9\}$&\small $\{301,\dots,454\}$&\checkmark&\ding{53}\\
		\hline
		\hline
		{$\{1,2,4,5,7,8\}$}&\small $\{603,\dots,753\}$&\checkmark&\ding{53}\\
		\hline
		\hline
		$\{0,2,3,4,5,9\}$&\small $\{603,\dots,653,754,\dots,853\}$&\checkmark&\ding{53}\\
		\hline
		\hline
		$\{1,2,3,4,5,6,7,9\}$&\small $\{854,\dots,1003\}$&\checkmark&\checkmark\\
		\hline
		\hline
		$\{0,2,3,4,5,6,8,9\}$&\small $\{1004,\dots,1154\}$&\checkmark&\checkmark\\
		\hline
		\hline
		$\{0,1,2,3,6,7,8\}$&\small $\{1155,\dots,1305\}$&\checkmark&\checkmark\\
		\hline
		\hline
		$\{0,3,4,6,7,8\}$&\small $\{1306,\dots,1457\}$&\checkmark&\checkmark\\
		\hline
	\end{tabular}
	\caption{Clustering results on dataset D10.}
	\label{tab:D10Report}
\end{table}
One example of a good performance of FIRES is the one illustrated in the previous section Figure~\ref{fig:bestResult} and Figure~\ref{fig:groundTruth}. By running SUBCLU on the same dataset, all clusters were found in the full relevant subspaces.

\section{Performance Evaluation and Comparison}\raggedbottom

\subsection{Quality}
To evaluate the clustering performed by SUBCLU and FIRES, we implemented some of the evaluation measures presented in the paper \citep{10.1145/2063576.2063774}. All these external evaluation measures work on comparing a resulting clustering to a known ground truth.

First, the F1-measure, which evaluates how good and precisely the true clusters are represented by the found clusters. A good representation is achieved when all true clusters are found and each found cluster $c_{found}$ has as many points as possible in common with a true cluster $c_{true}$ and as few incorrectly assigned points as possible. To cover both these aspects of a cluster representation, the F1-measure employs two other external performance metrics called recall and precision:
\begin{equation}
recall(c_{found},c_{true}) = \frac{TP}{TP + FN} = \frac{|c_{found} \cap c_{true}|}{|c_{true}|}.
\end{equation}

\begin{equation}
precision(c_{found},c_{true}) = \frac{TP}{TP + FP} = \frac{|c_{found} \cap c_{true}|}{|c_{found}|}.
\end{equation}

and computes the harmonic mean of them as follows:
\begin{equation}
F1(c_{found},c_{true}) = \frac{2\cdot recall(c_{found},c_{true})\cdot precision(c_{found},c_{true})}{recall(c_{found},c_{true}) + precision(c_{found},c_{true})}.
\end{equation}

The overall F1-measure of two clusterings $C_{found}$ and $C_{true}$ is then the mean of all maximum F1-measures of all clusters. We can differentiate between three computations of the overall F1-measure, depending on which clusters we are considering. If we are iterating over the true clusters, then we call it F1-Recall:
\begin{equation}
F1^{R}(C_{true},C_{found}) = \frac{1}{|C_{true}|} \sum _{c_{i}\in C_{true}} \max _{c_{j}\in C_{found}} \{F1(c_{i},c_{j})\}.
\end{equation}

Which can be seen as the recall but at the clustering level. As we can see, we are only considering found clusters that maximize the formula. Therefore, all redundant subspace clusters that are found in lower-dimensional projections and produced by a subspace clustering algorithm are not affecting the clustering quality negatively as they should do. On the other hand, iterating over found clusters yields the F1-Precision:
\begin{equation}
F1^{P}(C_{found},C_{true}) = \frac{1}{|C_{found}|} \sum _{c_{i}\in C_{found}} \max _{c_{j}\in C_{true}} \{F1(c_{i},c_{j})\}.
\end{equation}

Which is like precision but at the clustering level. Contrary to F1-Recall, F1-Precision is redundancy-aware, but since we are skipping true subspace clusters that are not a perfect match of found clusters with respect to the F1-measure, F1-Precision is not identification-aware because missing true clusters are not covered and are not affecting the quality.

A third form of the overall F1-measure can be computed by the mean of the F1-measures of true clusters and merged sets of found clusters. This measure is called F1-Merge and defined as follows:
\begin{equation}
\begin{multlined}
F1^{M}(C_{found},C_{true}) = \frac{1}{|C_{true}|} \sum _{c_{i}\in C_{true}} F1(c_{i},\acute{c}), \textrm{where}\\
\acute{c} = \bigcup _{c_{j}\in C_{found}} \Biggl\{c_{j} \mid \frac{|c_{j} \cap c_{i}|}{|c_{i}|} \geq \frac{|c_{j} \cap c_{t}|}{|c_{t}|} \forall  c_{t} \in C_{true} \Biggl\}.
\end{multlined}
\end{equation}

So, for each true cluster we merge all found clusters that share their best matching true clusters. Since we are iterating over true clusters, the F1-Merge measure is identification-aware but it is not redundancy-aware because found clusters in lower-dimensional projections are merged together with their higher-dimensional subspace clusters. In addition, merging found clusters together does not penalize split found clusters that represent one true cluster.

All three forms of the F1-measure are adapted from variants used to evaluate classification tasks and are not developed to evaluate subspace clustering. They are object-based because they work on comparing clusters of two clusterings based on data objects within the clusters. However, they are not subspace-aware since they are computed without regard to information about the subspaces in which the clusters exist. So, finding a true cluster in a different subspace than the true desired one will not affect the quality score. Therefore, high quality scores need not necessarily mean good subspace clustering. 

Subspace-aware quality measures are more adequate to the problem since we do not just want to check if true clusters are found but also if they are detected in the right subspaces.

The key idea that enables subspace-aware evaluations is to consider every data object as a set of different subobjects(also called micro-objects) rather than  one object in full-dimensional feature space. So, in a dataset with n objects and d dimensions, there will be n$\times$d micro-objects. A subspace cluster is then represented as a subset of these micro-objects. For example, let $c = \{1, 2, 3\}$ be a cluster in subspace $\{c,e,f\}$, then we redefine $c$ as $c = \{o_{i,j} | i\in \{1,2,3\} \wedge j\in \{c,e,f\}\} =  \{o_{1c},o_{2c},o_{3c},o_{1e},o_{2e},o_{3e},o_{1f},o_{2f},o_{3f}\}$. As a result, we now have the subspace information. 
\begin{center}
\begin{tikzpicture}[thick,fill opacity=0.5]
\draw[step=0.5cm,black,very thin] (-1.5,-1) grid (2.5,2);
\node at (-0.75,+1.75) [text opacity=1] {a};
\node at (-0.25,+1.75) [text opacity=1]{b};
\node at (+0.25,+1.75) [text opacity=1]{c};
\node at (+0.75,+1.75) [text opacity=1]{e};
\node at (+1.25,+1.75) [text opacity=1]{f};
\node at (+1.75,+1.75) [text opacity=1]{g};
\node at (+2.25,+1.75) [text opacity=1]{h};
\node at (-1.25,+1.25) [text opacity=1]{0};
\node at (-1.25,+0.75) [text opacity=1]{1};
\node at (-1.25,+0.25) [text opacity=1]{2};
\node at (-1.25,-0.25) [text opacity=1]{3};
\node at (-1.25,-0.75) [text opacity=1]{4};
\fill[gray] (-1,-1) rectangle (-0.5,-0.5);
\fill[gray] (-1,-0.5) rectangle (-0.5,0);
\fill[gray] (-1,0) rectangle (-0.5,0.5);
\fill[gray] (+0,0) rectangle (0.5,0.5);
\fill[gray] (+0.5,1) rectangle (1,0.5);
\fill[gray] (+0.5,0) rectangle (1,0.5);
\fill[gray] (+1,0) rectangle (1.5,0.5);
\fill[gray] (+1,0.5) rectangle (1.5,1);
\fill[gray] (0,0.5) rectangle (0.5,1);
\fill[gray] (+0,-0.5) rectangle (0.5,0);
\fill[gray] (+0.5,-0.5) rectangle (1,0);
\fill[gray] (+1,-0.5) rectangle (1.5,0);
\node at (+0.25,+0.75) [text opacity=1]{$o_{1c}$};
\node at (+0.25,+0.25) [text opacity=1]{$o_{2c}$};
\node at (+0.25,-0.25) [text opacity=1]{$o_{3c}$};
\node at (+0.75,+0.75) [text opacity=1]{$o_{1e}$};
\node at (+0.75,+0.25) [text opacity=1]{$o_{2e}$};
\node at (+0.75,-0.25) [text opacity=1]{$o_{3e}$};
\node at (+1.25,+0.71) [text opacity=1]{$o_{1f}$};
\node at (+1.25,+0.21) [text opacity=1]{$o_{2f}$};
\node at (+1.25,-0.28) [text opacity=1]{$o_{3f}$};
\end{tikzpicture}
\end{center}

The first subspace-aware evaluation measure we are going to look at is called RNIA (relative non-intersecting area). RNIA considers two clustering as finite sets of micro-objects and measures their similarity as follows:
\begin{equation} \label{eq:8}
RNIA(C_{found},C_{true}) = \frac{|U| - |I|}{|U|}.
\end{equation}

Where $U$ denotes the union of micro-objects of both clusterings and $I$ the intersection of their micro-objects. A good similarity is reached by a high hit rate and a low false positive rate of micro-objects. But this formula needs to be adjusted to cover micro-objects that belong to more than one subspace cluster due to simultaneously overlapping clusters and subspaces. A solution that redefines $U$ and $I$ was introduced in \citep{10.1109/TKDE.2006.106}, where $U = \sum_{i,j} \max(count_{true}(o_{i,j}), count_{found}(o_{i,j}))$ and $I = \sum_{i,j} \min(count_{true}(o_{i,j}), count_{found}(o_{i,j}))$ and $count$ the number of appearance in different subspace clusters. Briefly explained, each micro-object that is shared by different clusters will be dissolved and treated as several new unique micro-objects that share the same position in the matrix.

The best result is $RNIA = 0$ when both clusterings are equal $U = I$ and the worst result is $RNIA = 1$ when there is no intersection at all between true and found micro-objects.

Since the RNIA measure works at micro-object level and not on cluster level, neither clusters that include more than one class label nor splits and merges of found clusters are penalized.

To keep it simple for readers and similar to other quality scores, we compute $1 - RNIA$ so that 0 is the worst quality and 1 is the best.

Another subspace-aware evaluation measure is called CE (clustering error). CE functions similar to RNIA but it overcomes the inability of RNIA to detect splits and merges of found clusters by taking cluster-coverage into account. So, Instead of comparing true micro-objects with found micro-objects all at once without regard to which subspace clusters the micro-objects belong, CE searches for each true subspace cluster its best matching found cluster that has the most intersection of micro-objects with it and for each found subspace cluster its best matching true cluster. No true cluster is assigned to more than one found cluster and vice versa. This assignment should be chosen in such a way that it maximizes $D_{max}$ the sum over the intersections of each selected pair. The overall CE measure is then defined as follows:
\begin{equation} \label{eq:9}
CE(C_{found},C_{true}) = \frac{|U| - D_{max}}{|U|}.
\end{equation}

One way to find the perfect 1:1 assignment of clusters in relation to micro-objects intersection is to compute a square matrix of shape $max(|C_{found}|,|C_{true}|) \times max(|C_{found}|,|C_{true}|$) where 
an element represents the cardinality of the micro-objects' intersection of one true cluster and one found cluster. In case $|C_{found}| \neq |C_{true}|$ additional rows or columns are filled with zeros. Now, after computing the matching matrix, calculating $D_{max}$ would be like solving a maximum weight matching problem.

The best score $CE = 0$, is achieved when $|C_{found}| = |C_{true}|$ and for each true cluster there is a found cluster with the same micro-objects and vice versa. $U$ is computed as for RNIA and $D_{max}$ is not affected by overlapped subspaces and clusters.

As for the RNIA measure, we compute $1 - CE$ so that 0 is the worst quality and 1 is the best.

The last evaluation measure, called E4SC, was introduced for the first time in \citep{10.1145/2063576.2063774}. This novel quality measure has all the characteristics that a good subspace evaluation measure should have. It fulfills the subspace-awareness and the object-awareness criteria by using a subspace-aware version of the F1-measure in which recall and precision are redefined as follows:
\begin{equation}
recall_{sc}(c_{found},c_{true}) = \frac{|microObjects(c_{found}) \cap microObjects(c_{true})|}{|microObjects(c_{true})|}.
\end{equation}

\begin{equation}
precision_{sc}(c_{found},c_{true}) = \frac{|microObjects(c_{found}) \cap microObjects(c_{true})|}{|microObjects(c_{found})|}.
\end{equation}

And their harmonic mean:
\begin{equation}
F1_{sc}(c_{found},c_{true}) = \frac{2\cdot recall_{sc}(c_{found},c_{true})\cdot precision_{sc}(c_{found},c_{true})}{recall_{sc}(c_{found},c_{true}) + precision_{sc}(c_{found},c_{true})}.
\end{equation}

As for the non-subspace-aware overall F1-measure of two clusterings, we have $F1^{R}_{sc}$:

\begin{equation}
F1^{R}_{sc}(C_{true},C_{found}) = \frac{1}{|C_{true}|} \sum _{c_{i}\in C_{true}} \max _{c_{j}\in C_{found}} \{F1_{sc}(c_{i},c_{j})\}.
\end{equation}

And $F1^{P}_{sc}$:
\begin{equation}
F1^{P}_{sc}(C_{found},C_{true}) = \frac{1}{|C_{found}|} \sum _{c_{i}\in C_{found}} \max _{c_{j}\in C_{true}} \{F1_{sc}(c_{i},c_{j})\}.
\end{equation}
Which evaluates the representation of the true clusters.  

The symmetric evaluation measure E4SC is then defined as the harmonic mean of $F1^{R}_{sc}$ and $F1^{P}_{sc}$:
\begin{equation}
E4SC(C_{found},C_{true}) = \frac{2\cdot F1^{R}_{sc}(C_{true},C_{found})\cdot F1^{P}_{sc}(C_{found},C_{true})}{F1^{R}_{sc}(C_{true},C_{found}) + F1^{P}_{sc}(C_{found},C_{true})}.
\end{equation}

Missing true subspace clusters and redundant found clusters are detected and penalized by $F1^{R}_{sc}$ and $F1^{P}_{sc}$ respectively. 

We applied SUBCLU and FIRES with DBSCAN as a clustering method to both synthetic and real world datasets. But since we do not have any information about the true subspace clusters within the real world datasets, we adjust $C_{true}$ to represent the partitioning of the data objects in the full space, which is known due to provided class labels of all objects.

Three real world datasets were clustered and evaluated. First, a vowel dataset that contains information about sounds of the eleven steady state vowels of English spoken by multiple speakers. Second, a diabetes dataset which contains diagnostic measurements of diabetic and non-diabetic persons. Last, a glass dataset that contains examples of the chemical analysis of 7 different types of glass. All real world datasets are from the UCI machine learning repository \citep{Dua:2019}. In addition, we evaluated the clustering resulting from applying SUBCLU and FIRES to the synthetic datasets D05 and D10.

The best quality scores achieved for each dataset can be seen in Table~\ref{tab:quality} below:

\begin{table}[H]
\centering
\begin{tabular}{|c|c|c|c|c|c|c|c|}
	\hline
	&&\small F1-Recall&\small F1-Precision&\small F1-Merge&\small RNIA&\small CE&\small E4SC\\
	\hline
	\hline
	\multirow{2}{3.8em}{$\underset{528\times10}{\textrm{Vowel}}$}&\small SUBCLU&0.52&0.24&0.17&0&0&0.24\\
	&\small FIRES&0.28&0.24&0.17&0.52&0.10&0.13\\
	\hline
	\hline
	\multirow{2}{3.8em}{$\underset{768\times8}{\textrm{Diabetes}}$}&\small SUBCLU&0.65&0.55&0.65&0.01&0.01&0.46\\
	&\small FIRES&0.64&0.14&0.66&0.16&0.13&0.11\\
	\hline
	\hline
	\multirow{2}{3.8em}{$\underset{214\times9}{\textrm{Glass}}$}&\small SUBCLU&0.58&0.51&0.27&0&0&0.45\\
	&\small FIRES&0.40&0.55&0.27&0.54&0.23&0.23\\
	\hline
	\hline
	\multirow{2}{3.8em}{$\underset{1595\times5}{\textrm{D05}}$}&\small SUBCLU&0.99&0.84&0.20&0.15&0.15&0.78\\
	&\small FIRES&0.81&0.28&0.78&0.27&0.18&0.26\\
	\hline
	\hline
	\multirow{2}{3.8em}{$\underset{1595\times10}{\textrm{D10}}$}&\small SUBCLU&1&0.84&0.20&0.01&0.01&0.71\\
	&\small FIRES&0.91&0.34&0.68&0.12&0.10&0.29\\
	\hline
\end{tabular}
\caption{Quality comparison of SUBCLU and FIRES.}
\label{tab:quality}
\end{table}

\begin{table}[H]
	\centering
	\begin{tabular}{|c|c|c|c|c|c|}
		\hline
		&&\small F1-Merge&\small RNIA&\small CE&\small E4SC\\
		\hline
		\hline
		\multirow{2}{3.8em}{$\underset{990\times10}{\textrm{Vowel}}$}&\small SUBCLU&0.24&0.39&0.04&0.03\\
		&\small FIRES&0.16&0.14&0.02&0.03\\
		\hline
		\hline
		\multirow{2}{3.8em}{$\underset{768\times8}{\textrm{Diabetes}}$}&\small SUBCLU&0.74&0.01&0.01&0.46\\
		&\small FIRES&0.52&0.27&0.12&0.1\\
		\hline
		\hline
		\multirow{2}{3.8em}{$\underset{214\times9}{\textrm{Glass}}$}&\small SUBCLU&0.50&0.01&0.00&0.38\\
		&\small FIRES&0.30&0.45&0.21&0.24\\
		\hline
	\end{tabular}
	\caption{Quality comparison of SUBCLU and FIRES from \citep{10.1145/2063576.2063774}.}
	\label{tab:scoresFromPaper}
\end{table}

The evaluation scores regarding the Vowel dataset look noticeably dissimilar because the dataset has been updated and does not exist in the same shape as back then. Otherwise, the results look similar, taking into consideration that we do not know whether the authors did preprocess the datasets before clustering or not. We did not cluster the other real world datasets from the paper because we plotted them and did not find any clusters that could be detected.

To notice here is that SUBCLU has very low RNIA and CE scores. This can be explained by the huge number of redundant clusters produced by SUBCLU, which makes the cardinality of the union of micro-objects too high compared to the intersection  $|U| - |I| \approx |U|$ in Equation \ref{eq:8} and to $D_{max}$ in Equation \ref{eq:9} $|U| - D_{max} \approx |U|$. However, this redundancy leads to good cluster coverage, the reason why SUBCLU achieves good F1 scores. 

\subsection{Scalability}
Besides producing results of high quality, algorithms should also perform efficiently. Since we are dealing with high-dimensional data, it is expected for subspace clustering algorithms to perform slower when increasing the data dimensionality or the number of data objects, but also other factors like increasing the number of subspace clusters or their dimensionalities would affect the runtime of a subspace clustering algorithm.

For the following experiments, we chose DBSCAN as a clustering method for FIRES and we measured the runtime of all steps. 

First, to test the behavior with respect to increasing the dimensionality of the dataset, we applied both algorithms to six datasets having 5, 10, 15, 20, 25, 50 dimensions respectively. Each dataset consists of about 1600 points and 10 subspace clusters. The dimensionalities of the subspace clusters increase with increasing the dimensionality of the dataset. The results can be seen in Figure~\ref{fig:cluster_xd_xn_10sc}. In addition, to see the impact of the number of subspace clusters on the runtime of both algorithms, we redid the experiment on six other datasets with the same data dimensionalities but with only 1 subspace cluster per dataset. The results are illustrated in Figure~\ref{fig:cluster_xd_1000n_1sc} below:
\begin{figure}[H]
	\centering
	\begin{minipage}{0.5\textwidth}
		\centering
		\includegraphics[width=0.9\textwidth]{bilder/cluster_xd_xn_10sc}
		\caption{10 subspace clusters per dataset.}
		\label{fig:cluster_xd_xn_10sc}
	\end{minipage}\hfill
	\begin{minipage}{0.5\textwidth}
		\centering
		\includegraphics[width=0.9\textwidth]{bilder/cluster_xd_1000n_1sc}
		\caption{1 subspace clusters per dataset.}
		\label{fig:cluster_xd_1000n_1sc}
	\end{minipage}
\end{figure}
As we can see in Figure~\ref{fig:cluster_xd_xn_10sc} SUBCLU scales exponentially with respect to the data dimensionality and needed about an hour to cluster the dataset with 20 dimensions, the reason why we did not apply it to the other datasets with more dimensions. All studies like \citep{10.14778/1687627.1687770} and \citep{10.1145/2063576.2063774} that include SUBCLU have also noticed such inefficient performance. This is caused by the bottom-up heuristic used by SUBCLU to investigate possible subspaces. It is not clear how FIRES exactly scales because the number of runtime measurements is not enough and SUBCLU has too high runtimes compared to FIRES. However, it is noticeably faster, because it jumps from the 1-dimensional base-clusters directly to subspace clusters of maximal dimensionalities without going through the whole space of all possible subspaces.

The runtime drops significantly if there are fewer subspace clusters as shown in Figure~\ref{fig:cluster_xd_1000n_1sc}, but even with only one subspace cluster, SUBCLU still scales at least quadratic. 

To measure the time complexities of SUBCLU and FIRES with respect to the number of points in the dataset, we applied both on five datasets having between 1500--5500 data points, Figure~\ref{fig:cluster_10d_xn_10sc}. Each data set consists of 10 dimensions and 10 subspace clusters. Additionally, we redid the experiment with only 2 subspace cluster per dataset. The results are in Figure~\ref{fig:cluster_10d_xn_2sc}.
\begin{figure}[H]
	\centering
	\begin{minipage}{0.5\textwidth}
		\centering
		\includegraphics[width=0.9\textwidth]{bilder/cluster_10d_xn_10sc}
		\caption{10 subspace clusters per dataset.}
		\label{fig:cluster_10d_xn_10sc}
	\end{minipage}\hfill
	\begin{minipage}{0.5\textwidth}
		\centering
		\includegraphics[width=0.9\textwidth]{bilder/cluster_10d_xn_2sc}
		\caption{2 subspace clusters per dataset.}
		\label{fig:cluster_10d_xn_2sc}
	\end{minipage}
\end{figure}
Increasing the number of data points in the dataset does not affect the runtime as much as the data dimensionality. However, if this number is combined with a certain dimensionality, then it could lead to a very high runtime of SUBCLU. For example, in \citep{10.14778/1687627.1687770} evaluating the clustering of the Pendigits dataset 7494$\times$416 by SUBCLU was not possible since SUBCLU did not even finish running. This is the reason why we chose 10 to be the datasets dimensionality in this experiment.

The increment in runtime is caused by the range queries used by DBSCAN to find neighbors, since the more data points there is, the more time needed to calculate the distances to other points. But since SUBCLU explores many more subspaces than FIRES and applies DBSCAN on the data points in each of these subspaces, it is more affected, as Figure~\ref{fig:cluster_10d_xn_10sc} and Figure~\ref{fig:cluster_10d_xn_2sc} show. 

Let us note, when clustering by FIRES we could not find the right hyperparameters for the datasets with 4500 and 5500 points and 10 subspace clusters. This is why the runtime drops in Figure~\ref{fig:cluster_10d_xn_10sc} while we are expecting it to increase.



\section{Conclusion and Future Work}\raggedbottom
It is known in machine learning that it is not simple to say an algorithm is definitively better than another one, because algorithms could behave differently on different kinds of problems. But based on the experiments we did, we can make a judgment on both algorithms. 

Regarding SUBCLU, we can say that it produces clustering results of good quality. However, the efficiency of the algorithm is degraded at the expense of the quality of the clustering. Therefore, SUBCLU is biased towards low dimensionality and is not feasible for solving the problem for which it was developed, since we are aiming at clustering high-dimensional data with up to several hundreds of dimensions. 

FIRES, on the other hand, is much faster. Therefore, it is more qualified for the problem of clustering high dimensional data. However, FIRES is difficult to use due to its strong dependency on its hyperparameters, which are not few and not easy to tune, which makes it most likely for clustering results of FIRES to not live up to the user's expectations.

We think both algorithms can be further improved. For example, by trying to modify FIRES in a way that makes it depend on fewer hyperparameters. For SUBCLU, one could try modifying the algorithm to consider fewer subspaces in order to improve the algorithm's efficiency. A redundancy removal technique would be beneficial too. Developing a fuzzy variant of SUBCLU and FIRES could be done in future research.

%%%%%%%%%%%%%%%%%%%%%%%%%%%%%%%%%%%%%%%%%%%%%%%%%%%%%%%%%%%%%%%%%%%%%%%%
%%%% ENDE TEXTTEIL %%%%%%%%%%%%%%%%%%%%%%%%%%%%%%%%%%%%%%%%%%%%%%%%%%%%%
%%%%%%%%%%%%%%%%%%%%%%%%%%%%%%%%%%%%%%%%%%%%%%%%%%%%%%%%%%%%%%%%%%%%%%%%

\clearpage

% Entfernen Sie das Kommentar aus der nachfolgenden Zeile, falls Sie einen Anhang in der Arbeit verwenden wollen. Beachten Sie, dass Sie sich im Verlauf der Arbeit mit \ref{...} (z.B. \ref{anhang:zusatz1}) auf den Anhang beziehen.
%\newpage
\appendix
\section{Anhang}

\subsection*{Zusatzteil 1} \label{anhang:zusatz1}

Dies ist ein Anhang.

\clearpage


\ifthenelse{\boolean{\biber}}{ %with biber do
	\DeclareNameAlias{sortname}{first-last}
	\printbibliography[heading=bibintoc, title=\references]
}{ %without biber do
	\bibliography{references}
	\bibliographystyle{alphadin}
}
%\vspace*{\fill}

\clearpage

\listoffigures

\listoftables

%\pagebreak

%\printindex
\end{document}
