\section{Overview of Related Work}\raggedbottom

To the best of our knowledge, the authors of SUBCLU and FIRES have not published any implementation of both the algorithms, which urged all researchers that have an interest in investigating the two algorithms to implement them by writing their own programs.
In addition, scientific researches that include SUBCLU or FIRES as part of some specific studies have only presented the end results of their study in relation to SUBCLU or FIRES, but no implementations were provided as this was not the main topic of these researches.

An example implementation of SUBCLU in Java \href{https://github.com/elki-project/elki/blob/master/elki-clustering/src/main/java/elki/clustering/subspace/SUBCLU.java}{SUBCLU.java} can be found in the official github repository of ELKI (Environment for Developing KDD-Applications Supported by Index-Structures), a general framework for data mining. Another implementation of SUBCLU and FIRES in Java should be found in the OpenSubspace framework, which is an Open Source Framework for Evaluation and Exploration of Subspace Clustering Algorithms in WEKA, but at the time of writing this thesis, the webpage of the project \href{http://dme.rwth-aachen.de/en/opensubspace}{http://dme.rwth-aachen.de/en/opensubspace} do not exist anymore.

There are some papers that have done comparative studies on both the algorithms. For example, in the FIRES \citep{fires} paper itself, after developing the algorithm, the authors tested its effectiveness against SUBCLU \citep{subclu} by applying both to synthetic data and then measuring the clustering quality, by first checking whether the hidden subspace clusters are discovered or not, and second by measuring how precise is the clustering in relation to noise. 
The authors then compared the scalability of SUBCLU and FIRES by measuring the runtime of both algorithms while increasing the number of points in the dataset, the data dimensionality and the dimensionality of a subspace cluster. These experiments have shown that FIRES is more accurate and more efficient. In addition, FIRES was more successfully applied on real-world high-dimensional gene expression data, unlike SUBCLU, which failed to detect some subspace clusters and caused memory overflow due to producing a high number of candidate subspaces.

Another evaluation of both algorithms can be found in the paper titled “Evaluating Clustering in Subspace Projections of High Dimensional Data” \citep{10.14778/1687627.1687770}. This publication aimed to present a fair and thorough evaluation and comparison of different developed subspace clustering paradigms by applying evaluation criteria thoughtfully so that all paradigms are tested under impartial experimental conditions. 

The authors did object-based and object- and subspace-based external evaluation measures with respect to the ground truth provided by synthetic data. Object-based measures included measuring the entropy (the purity of the found clusters with respect to the actual hidden subspace clusters in the data), F1 (to evaluate how well the hidden clusters are represented by the found clusters), accuracy (to evaluate how good the ground truth is generalized by the identified clusters).

Whereas object- and subspace-based evaluation measures included evaluation performance parameters in relation to the relevant dimensions of the subspace clusters. For these measurements, each data object in a D-dimensional database was partitioned into D different subobjects, also called micro-objects, so that subspace clusters can be represented as subsets of micro-objects. The first included an evaluation approach called RNIA (relative non-intersecting area) that measures how well the subobjects of the ground truth are covered by the found subobjects. The second evaluation approach is called CE (clustering error) which fulfills the same quality criteria as RNIA, but penalizes clusters which split up into several smaller ones.

The performance evaluation results show that SUBCLU detects much more subspace clusters than FIRES does and therefore, SUBCLU has a better coverage score but worse scores with respect to entropy, RNIA and CE. SUBCLU has also  achieved better F1 and accuracy scores. However, evaluating the scalability of SUBCLU and FIRES with respect to data dimensionality and database size, shows that SUBCLU could not compete against FIRES, as the runtime of SUBCLU could be extremely high, up to several days, if the data dimensionality or size above some certain numbers. The authors used their open source framework \textit{OpenSubspace} for the analysis. All implementations, datasets and evaluation measures should actually be available on the website  \href{http://dme.rwth-aachen.de/OpenSubspace/evaluation}{http://dme.rwth-aachen.de/OpenSubspace/evaluation} as mentioned in the article.

Another comparison of FIRES and SUBCLU using the novel evaluation measure E4SC can be seen in \citep{10.1145/2063576.2063774}. The E4SC measure has all the characteristics of a complete external subspace evaluation measure since it fulfills all general quality criteria for subspace clustering measures.
