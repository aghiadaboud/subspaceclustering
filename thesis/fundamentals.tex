\section{Fundamentals}\raggedbottom

Although there are various approaches that address the problem of subspace clustering, the fundamental terms are globally defined. Three of these terms are explained as follows:

\textbf{Definition 1 (Subspace)} Let $DB$ be a data set and $A$ be the set of all attributes spanning the feature space of DB. Any subset $S \subseteq A$ of the original feature space is a subspace. 

\textbf{Definition 2 (Subspace Cluster)} Let $n$ objects be in $DB$. A group of $m \leq n $ objects that are similar in all dimensions of a subspace $S$ is called a subspace cluster. 

\textbf{Definition 3 (Dimensionality of Subspace Cluster)} The number of dimensions $(|S|)$ forming the subspace in which the cluster exists.

\textbf{Definition 4 (Overlapping Subspace Clusters)} Let $C_{i}$ be a cluster in subspace $S_{i}$ and $C_{j}$ a cluster in subspace $S_{j}$. We call $C_{i}$ and $C_{j}$ overlapped, if ($C_{i} \neq C_{j}$), $(C_{i} \cap C_{j}) \neq \emptyset$ and $(S_{i} \cap S_{j}) = \emptyset$.

\textbf{Definition 5 (Overlapping Subspaces)} Let $C_{i}$ be a cluster in subspace $S_{i}$ and $C_{j}$ a cluster in subspace $S_{j}$. We call $S_{i}$ and $S_{j}$ overlapped, if ($S_{i} \neq S_{j}$), $(S_{i} \cap S_{j}) \neq \emptyset$ and $(C_{i} \cap C_{j}) = \emptyset$.

\textbf{Definition 6 (Simultaneously Overlapping Clusters and Subspaces)} Let $C_{i}$ be a cluster in subspace $S_{i}$ and $C_{j} \neq C_{i}$ a cluster in subspace $S_{j} \neq S_{i}$. We call $C_{i}$, $C_{j}$ and $S_{i}$, $S_{j}$ overlapped, if $(C_{i} \cap C_{j}) \neq \emptyset$ and $(S_{i} \cap S_{j}) \neq \emptyset$.

Any other not general technical concepts that reflect the understanding of authors will be defined in later sections.