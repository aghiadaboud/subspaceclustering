%%% Die folgende Zeile nicht ändern!
\section*{\ifthenelse{\equal{\sprache}{deutsch}}{Zusammenfassung}{Abstract}}\raggedbottom
%%% Zusammenfassung:

Subspace clustering has been developed to meet the requirements for clustering high-dimensional data at a time when traditional clustering methods were not able to solve the problem properly. It is a powerful method that mines hidden clusters in subspaces of the original feature space by combining two tasks, subspace search and clustering.

Subspace clustering has been successfully applied in fields like gene expression analysis, where some clusters can only be present in locally relevant subsets of dimensions.

There are many subspace clustering algorithms, but in this thesis we will discuss only SUBCLU \citep{subclu} and FIRES \citep{fires} in detail, two density-based subspace clustering algorithms. We will illustrate an example of clustering done by our implementation of both algorithms and we will evaluate their scalability and the quality of their clustering results.

In addition, we will talk about the reasons for which subspace clustering was developed. Then we will mention some applications where it can be employed and some alternatives for the method. We will also list all different classes of subspace clustering approaches and explain in which aspects these approaches can differ.